%
%%% LYHENTEET JA MÄÄRITELMÄT (+ käännökset?)
%
% Tässä tiedostossa määritellään tekstissä kätetyt lyhenteet
%
% LaTeX rakentaa näistä automaattisesti sivun, jossa esitellään kaikki
% dokumentissa esiintyvät lyhenteet (akronyymit). Valmiissa pdf:ssä
% jokainen lyhenne on myös linkki sen määritelmään, eli linkkiä 
% klikkaamalla pääsee suoraan sen määritelmään. 

\makeglossaries

    \newacronym{gdpr}{GDPR}{General Data Protection Regulation}
    \newacronym{crm}{CRM}{Customer Relationship Management}
    \newacronym{erp}{ERP}{Enterprise Resource Planning}
    \newacronym{ups}{UPS}{Uninterruptible Power Supply}
    \newacronym{api}{API}{Application Programming Interface}
    \newacronym{crud}{CRUD}{create, read, update, and delete}

    \newglossaryentry{gdprg}
    {
    name={General Data Protection Regulation (GDPR)},
	description={General Data Protection Regulation on Euroopan unionin asettama direktiivi tietosuojakäytännöistä},
	first={General Data Protection Regulation (GDPR)},
	text={GDPR}
    }
    \glsadd{gdprg}

    \newglossaryentry{crmg}
    {
	name={Customer Relationship Management (CRM)},
	description={Customer Relationship Management eli asiakkuudenhallintajärjestelmä \cite{crm, crm2}},
	first={Customer Relationship Management (CRM)},
	text={CRM}
    }
    \glsadd{crmg}

    \newglossaryentry{erpg}
    {
    name={Enterprise Resource Planning (ERP)},
    description={Toiminnanohjausjärjestelmä on yrityksen tietojärjestelmä, joka integroi eri toimintoja, esimerkiksi tuotantoa, jakelua, varastonhallintaa, laskutusta ja kirjanpitoa \cite{erp}},
    first={Enterprise Resource Planning (ERP)},
    text={ERP}
    }
    \glsadd{erpg}

    \newglossaryentry{upsg}
    {
    name={Uninterruptible Power Supply (UPS)},
    description={Järjestelmä, joka takaa tasaisen virransyötön, vaikka järjestelmän ulkopuolinen virransyöttö katkeaisi tai virransyötössä ilmenee joku muu häiriö. \cite{ups}},
    first={Uninterruptible Power Supply (UPS)},
    text={UPS}
    }
    \glsadd{upsg}

    \newglossaryentry{apig}
    {
    name={Application Programming Interface (API)},
    description={määritelmä, jonka mukaan eri ohjelmat voivat vaihtaa tietoja eli keskustella keskenään. \cite{api}},
    first={Application Programming Interface (API)},
    text={API}
    }
    \glsadd{apig}

    \newglossaryentry{crudg}
    {
    name={Create, Read, Update, and Delete (CRUD)},
    description={Tietojenkäsittelyn neljä perustoimintoa - luo, lue, päivitä ja poista},
    first={create, read, update, and delete (CRUD)},
    text={CRUD}
    }
    \glsadd{crudg}

    \newglossaryentry{kayttajag}
    {
    name={Käyttäjä},
    description={Henkilö tai taho, joka jollain tavalla hyödyntää järjestelmää},
    }
    \glsadd{kayttajag}

    \newglossaryentry{ulkoinenkayttajag}
    {
    name={Ulkoinen käyttäjä},
    description={Yksityis- tai yritysasiakas}
    }
    \glsadd{ulkoinenkayttajag}

    \newglossaryentry{sisainenkayttajag}
    {
    name={Sisäinen käyttäjä},
    description={Kaikki konsernin sisällä toimivat tahot, jotka hyödyntävät järjestelmää. (esim. myynti- ja markkinointiosasto)}
    }
    \glsadd{sisainenkayttajag}

    \newglossaryentry{paakayttajag}
    {
    name={Pääkäyttäjä},
    description={Käyttäjätili, jolla on korkein mahdollinen oikeustaso järjestelmään}
    }
    \glsadd{paakayttajag}

    \newglossaryentry{tietokantag}
    {
    name={Tietokanta},
    description={Tietojen keräämiseen ja järjestämiseen käytettävä työkalu, johon voidaan tallentaa esimerkiksi henkilöihin, tuotteisiin tai tilauksiin liittyviä tietoja \cite{kurssi}}
    }
    \glsadd{tietokantag}

    \newglossaryentry{redundantg}
    {
    name={Reduntant},
    description={Varmistettu (kahdennettu) linkki, joka toimii varana, jos ensisijainen järjestelmä kaatuu \cite{kurssi}}
    }
    \glsadd{redundantg}

    \newglossaryentry{virtualg}
    {
    name={Virtualisointi},
    description={Fyysisten laitteiden mallinnus virtuaaliseksi säilyttäen sama toiminnallisuus kuin alkuperäiselläkin laitteella.}
    }
    \glsadd{virtualg}

    \newglossaryentry{accesspolicyg}
    {
    name={Access Policy - Pääsynhallinta},
    description={Selvästi luokitellut ja rajatut käyttäjäryhmät, joille on tarkasti määritellyt oikeudet \cite{sommerville}}
    }
    \glsadd{accesspolicyg}

        


