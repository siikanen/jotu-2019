%
%%% LYHENTEET JA MÄÄRITELMÄT (+ käännökset?)
%
% Tässä tiedostossa määritellään tekstissä kätetyt lyhenteet
%
% LaTeX rakentaa näistä automaattisesti sivun, jossa esitellään kaikki
% dokumentissa esiintyvät lyhenteet (akronyymit). Valmiissa pdf:ssä
% jokainen lyhenne on myös linkki sen määritelmään, eli linkkiä 
% klikkaamalla pääsee suoraan sen määritelmään. 

\makeglossaries

    \newacronym{gdpr}{GDPR}{General Data Protection Regulation}
    \newacronym{crm}{CRM}{Customer Relationship Management}



    \newglossaryentry{gdprg}
    {
        name={General Data Protection Regulation (GDPR)},
	description={General Data Protection Regulation on Euroopan unionin asettama direktiivi tietosuojakäytännöistä},
	first={General Data Protection Regulation (GDPR)},
	text={GDPR}
    }
    \glsadd{gdprg}

    \newglossaryentry{crmg}
    {
	name={Customer Relationship Management (CRM)},
	description={Customer Relationship Management eli asiakkuudenhallintajärjestelmä},
	first={Customer Relationship Management (CRM)},
	text={API}
    }
    \glsadd{crmg}

