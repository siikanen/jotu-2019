%
%%% LYHENTEET JA MÄÄRITELMÄT (+ käännökset?)
%
% Tässä tiedostossa määritellään tekstissä kätetyt lyhenteet
%
% LaTeX rakentaa näistä automaattisesti sivun, jossa esitellään kaikki
% dokumentissa esiintyvät lyhenteet (akronyymit). Valmiissa pdf:ssä
% jokainen lyhenne on myös linkki sen määritelmään, eli linkkiä 
% klikkaamalla pääsee suoraan sen määritelmään. 

\makeglossaries

    \newacronym{gdpr}{GDPR}{General Data Protection Regulation}
    \newacronym{crm}{CRM}{Customer Relationship Management}
    \newacronym{erp}{ERP}{Enterprise Resource Planning}


    \newglossaryentry{gdprg}
    {
    name={General Data Protection Regulation (GDPR)},
	description={General Data Protection Regulation on Euroopan unionin asettama direktiivi tietosuojakäytännöistä},
	first={General Data Protection Regulation (GDPR)},
	text={GDPR}
    }
    \glsadd{gdprg}

    \newglossaryentry{crmg}
    {
	name={Customer Relationship Management (CRM)},
	description={Customer Relationship Management eli asiakkuudenhallintajärjestelmä},
	first={Customer Relationship Management (CRM)},
	text={CRM}
    }
    \glsadd{crmg}

    \newglossaryentry{erpg}
    {
    name={Enterprise Resource Planning (ERP)},
    description={ toiminnanohjausjärjestelmä on yrityksen tietojärjestelmä, joka integroi eri toimintoja, esimerkiksi tuotantoa, jakelua, varastonhallintaa, laskutusta ja kirjanpitoa}
    first={Enterprise Resource Planning (ERP)}
    text={ERP}
    }

    \newglossaryentry{kayttajag}
    {
    name={Käyttäjä},
    description={Henkilö tai taho, joka jollain tavalla hyödyntää järjestelmää}
    }
    \glsadd{kayttajag}

    \newglossaryentry{ulkoinenkayttajag}
    {
    name={Ulkoinen käyttäjä},
    description={Yksityis- tai yritysasiakas}
    }
    \glsadd{ulkoinenkayttajag}

    \newglossaryentry{sisainenkayttajag}
    {
    name={Sisäinen käyttäjäg},
    description={Kaikki konsernin sisällä toimivat tahot, jotka hyödyntävät järjestelmää. (esim. myynti- ja markkinointiosasto)}
    }
    \glsadd{sisainenkayttajag}

    \newglossaryentry{paakayttajag}
    {
    name={Pääkäyttäjä}
    description={Henkilö, joka käyttää järjestelmää säännöllisesti ja vastaa sen sisällön tuottamisesta}
    }
    \glsadd{paakayttajag}


