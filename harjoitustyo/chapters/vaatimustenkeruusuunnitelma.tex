\chapter{Vaatimustenkeruusuunnitelma} % Itse luvun otsikko. Huom ei numeroa!
\label{keruu} % Tähän kappaleeseen voi viitata \ref{keruu}
\thispagestyle{fancy} % Tarvitaan, jotta header/footer näkyvät otsikkosivuilla

Osana ensimmäistä vaihetta suunnitellaan, miten vaatimukset saadaan kerättyä tarpeellisilta sidosryhmiltä ja tehdään sidosryhmäanalyysi. 

\section{Taustatilanne}  % Kappaleen otsikko

    Miten asiakashallinta on hoidettu yrityksessä aikaisemmin? Miten se on toiminut ja mitkä ovat keskeisimmät puutteet?

    % Täytetekstiä, poista kun lisäät oikeaa sisältöä
    \lipsum

\subsection{Alaotsikko}

    % Täytetekstiä, poista kun lisäät oikeaa sisältöä
    \lipsum[2]


\section{Nykyisen dokumentaation analyysi}

    Mitä dokumentaatiota tällä hetkellä on tarjolla(kehyskertomus)? Mitä sen perusteella tiedetään? Myös kilpailevat tuotteet.

    \lipsum[1]

\section{Käytettävät metodit}

Vaatimusten keräyksessä käytettävät menetelmät. Ks. luentomateriaali. Tämä on turhaa täytetekstiä, jotta nähtäisiin miltä dokkari näyttää:

    \begin{itemize}
        \item Kalja
        \item Kalja
        \item Kalja
        \item Kalja    
    \end{itemize}

    \lipsum[3]


\section{Sidosryhmäanalyysi}

    Mikä se sidosryhmäanalyysi nyt sitten on tähän kohtaan ja mitä seuraavissa kpl käsitellään

    \subsection{Iso kuva (järjestelmä, ohjelmisto\-projekti ja sen ympä\-ristö)}

    Ks. esimerkkidokumentaatiosta Iso\_kuva.pdf(huom, kyseessä on esimerkki formaatista, esimerkin sisältö on erilaiseen tilanteeseen ja projektiin liittyvää.Huomioi myös, että esimerkkimateriaalit eivät ole keskenään samoista projekteista!)Sidosryhmät järkevästi luokiteltuna. Mahdolliset muut järjestelmät. Muut liittyvät asiat.

    \subsection{Analyysin tulokset}
    
    Sidosryhmäanalyysin tulokset, formaattina esim. Excel.Ks. esimerkkidokumentaatiosta Sidosryhmäanalyysin tulokset.pdf(huom, kyseessä on esimerkki formaatista, esimerkin sisältö on erilaiseen tilanteeseen ja projektiin liittyvää.Huomioi myös, että esimerkkimateriaalit eivät ole keskenään samoista projekteista!)Sidosryhmien luokittelu, rooli, millä menetelmillä sidosryhmältä kerätään vaatimukset, sidosryhmän perustelu (miksi tarvitsemme tätä sidosryhmää?) ja tarvittava osallistuminen (milloin sidosryhmää tarvitaan). Lisäksi kategorisointi siitä, minkälaisia/mihin liittyviä vaatimuksia sidosryhmältä odotetaan saatavaksi.

\section{Alustavat havaitutvaatimukset ja luokittelu}

    Annetun kehyskertomuksen pohjalta tunnistetut vaatimukset. Suunniteltu vaatimusten luokittelu, lähde, priorisointi ja vaatimusten koontiin sopiva taulukkopohja. Ks. esimerkkidokumentaatiosta Vaatimukset\_esimerkki.pdf(huom, kyseessä on esimerkki formaatista, esimerkin sisältö on erilaiseen tilanteeseen ja projektiin liittyvää.Huomioi myös, että esimerkkimateriaalit eivät ole keskenään samoista projekteista!).

\section{Vaatimusten keruuprojektin aikataulu (Gantt)}

    Vaatimusten keruun aikataulusuunnitelma sidosryhmille. Kuka tekee, mitä tekee ja koska. Formaattina Gantt-kaavio. Ks. esimerkkidokumentaatiosta Vaatimusten keruun suunnitelma.pdf(huom, kyseessä on esimerkki formaatista, esimerkin sisältö on erilaiseen tilanteeseen ja projektiin liittyvää.Huomioi myös, että esimerkkimateriaalit eivät ole keskenään samoista projekteista!).
