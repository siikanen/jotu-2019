\chapter{Vaatimustenkeruusuunnitelma} % Itse luvun otsikko. Huom ei numeroa!
\label{keruu} % Tähän kappaleeseen voi viitata \ref{keruu}
\thispagestyle{fancy} % Tarvitaan, jotta header/footer näkyvät otsikkosivuilla

\lipsum[1]

\section{Otsikko}  % Kappaleen otsikko

% Täytetekstiä, poista kun lisäät oikeaa sisältöä
\lipsum

\subsection{Alaotsikko}

% Täytetekstiä, poista kun lisäät oikeaa sisältöä
\lipsum[1-3]


\section{Nykyisen dokumentaation analyysi}
Tässä kappaleessa käsitellään jo olemassa olevia dokumentaatioita, ja mitä vaatimuksia asiakastietojärjestelmälle voidaan niiden perusteella määrittää.



\subsection{Kehyskertomuksen analyysi}
Con-Salting Oy on kerännyt alustavia vaatimuksia kehyskertomukseen. Kehyskertomuksessa määritetään asiakastietojärjestelmälle alustavia vaatimuksia.
Kehyskertomuksen perusteella voidaan määrittää sidosryhmiä, jotka tulee ottaa huomioon, ja joiden kanssa tulee kommunikoida vaatimuksia määriteltäessä.
Tällaisia sidosryhmiä ovat muun muassa yrityksen johto ja myynti- ja markkinointiosasto.
sidosryhmät ovat esittäneet toivomuksina esimerkiksi asiakastietojärjestelmän nopean toiminnan, käytettävyyden erilaisilta päätelaitteilta sekä asiakkaan 
automaattisen profiloinnin. 
    Kehyskertomuksessa ilmenee myös muita asioita jotka tulee ottaa huomioon:
Järjestelmän tulee olla yhteensopiva muiden yrityksen järjestelmien kanssa kuten yrityksen varastojärjestelmän kanssa.
Asiakastietojärjestelmän tulee myös noudattaa GDPR:ää(General Data Protection Regulation)
Järjestelmän tulee pystyä hoitaa joitain asioita automaattisesti, kuten suurostobonukset.

Nämä vaatimukset ovat kuitenkin ainoastaan suuntaan antavia, ja niitä tulee tarkentaa vaatimustenkeruussa. 





\section{Vaatimustenkeruu metodit}
Vaatimustenkartutusmetodit voidaan jakaa kahteen eri osa-alueeseen: Epäsuoriin- ja suoriin metodeihin.
Tässä kappaleessa käsitellään Megantin asiakastietojärjestelmän kehitystä varten käytettäviä metodeja.


\subsection{Suorat metodit}
Suorissa kartutustekniikoissa järjestelmän vaatimuksia kartoitetaan yhdessä sidosryhmien kansssa.
Käytämme seuraavia suorakartutustekniikoita:

Haastattelut
Järjestämme kahdentyyppisiä haastatteluja sidosryhmille. Toisessa haastattelussa haastateltavat vastaavaat ennaltamääriteltyihin kysymyksiin. 
Toisessa haastattelussa haastattelu toteutetaan avoimesti. Haastateltaville esitetään erilaisia kysymyksiä järjestelmän toiminnalisuuteen liittyen jolloin
haastateltavien kanssa voidaan vuorovaikutteisesti pohtia millainen järjestelmän tulisi olla.

Havainnointi
Lähetämme 2 kehitystiimiläistä vierailemaan  Meganttiin. Nämä tiimiläiset seuraavat päivän ajan Megantin työntekijöiden työtä, ja tekevät 
muistiinpanoja työntekijöiden tarpeista järjestelmään liittyen. He havainnoivat myös mitä puutteita ja vahvuuksia nykyisessä asiakastietojärjestelmässä on.
Käytämme havainnointia siitä syystä, että ihmisten on usein vaikea pukea arkipäivän työntekoa sanoiksi. Havainnoin avulla pääsemme eroon tästä haittatekijästä.


Ryhmätapaamiset 
Projektin aikana järjestetään tapaamisia sidosryhmien kanssa. Näissä tapaamisissa kehitystiimi kertovat asiakastietojärjestelmän nykytilasta, kuvailevat tilanteita 
johon sidosryhmät voivat vastata kuinka he haluavat järjestelmän toimivan kyseisissä tilanteissa.


\subsection{Epäsuorat metodit}
Epäsuorissa kartutustekniikoissa sidosryhmiin ei olla suorassa kontaktissa, vaan käytetään jo ennalta olevia tietoja.
Käytämme seuraavia epäsuorakartutustekniikoita:

Taustatutkimus 
Con-Salting Oy on kerännyt jo muutamia yrityksen vaatimuksia kehyskertomukseen. 
Näitä vaatimuksia ovat esimerkiksi asiakastietojen ylläpito, ja asiakkaiden profilointi.
Vaatimukset ovat kuitenkin löysästi määriteltyjä ja vaativat tarkennusta.

Kyselyt
Osalle järjestelmän sidosryhmistä järjestetään kyselyitä, jossa selvitetään heidän mielestään tärkeimpiä järjestelmän vaatimuksia.
Tällaisia sidosryhmiä ovat sisäiset käyttäjät, kuten yrityksen johto, asiakaspalvelu ja järjestelmän ylläpito henkilöstö.

Prototyypit
Sisäisille käyttäjille valmistetaan kahteen otteeseen prototyyppi järjestelmästä. Käyttäjät voivat siis testata järjestelmää aikaisessa vaiheessa ja antaa palautetta.