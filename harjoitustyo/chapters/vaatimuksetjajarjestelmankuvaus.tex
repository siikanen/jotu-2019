\chapter{Vaatimukset ja järjestelmän kuvaus} % Itse luvun otsikko. Huom ei numeroa!                                         santeri on ruma
\label{kuvaus} % Tähän kappaleeseen voi viitata \ref{kuvaus}
\thispagestyle{fancy} % Tarvitaan, jotta header/footer näkyvät otsikkosivuilla


\section{Mallintaminen}  % 3.1
    Tässä kappaleessa kuvaillaan kaksi käyttäjätarinaa. 


\subsection{K\äytt\ötapauskaavio(t)}

\subsubsection{Käyttäjätapaus 1: Käyttäjätietojen tarkastelu}

    Antti Asiakas haluaa tarkastella itseään koskevia tietoja sekä raportteja. Antin tulee valita etusivulta löytyvä
    \textit{Omat tiedot} -painike, jotta hän pääsee katsomaan omia tietojaan. 

    Täältä Antti voi:

    \begin{itemize}
        \item tarkastella ja muuttaa omia yhteystietojaan
        \item tarkastella itseään koskevia raportteja
        \item poistaa itsensä järjestelmästä
        \item kirjautua ulos
    \end{itemize}

    Jos Antti valitsee itseään koskevat raportit, järjestelmä esittää hänelle hänen oman ostohistoriansa,
    voimassa olevat alennukset sekä mahdolliset suurostobonukset.
    Jos Antilla ei ole aiempaa historiaa Megantin kanssa, järjestelmä ei tarjoa hänelle raporttia.

\subsubsection{Käyttäjätapaus 2: Personoidun tarjouksen\/tajouksien luonti ja lähettäminen}

    Järjestelmä ehdottaa Mikko Myyntitykille mahdollista alennusta koskien tiettyä asiakasryhmää.
    Mahdollinen alennus perustuu järjestelmän omiin sisäisiin statistiikkoihin ja algoritmeihin.
    Mikko voi hyväksyä, olla hyväksymättä alennusta, tai halutessaan määritellä tarjouksen itse.
    Jos Mikko syöttää järjestelmään alennuksen, joka on erittäin suuri, järjestelmä antaa hänelle varoituksen.
    Mikko voi myös tarkastaa, että järjestelmä ei luo alennuksia, jotka ovat liian suuria tai pieniä.
    Alennusta myönnettäessä määritellään sen suuruus, kesto, tuoteryhmät sekä asiakaskohderyhmä.

    Jos jonkinlainen alennus myönnetään, järjestelmä lähettää asiakasryhmälle sähköisen ilmoituksen alennuksesta (tarjouskirje).

    



\section{K\äytt\öliittymä}



\section{Vaatimukset}




\section{Ympäristö}



   