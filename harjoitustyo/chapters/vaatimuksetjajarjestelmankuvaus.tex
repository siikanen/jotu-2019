\chapter{Vaatimukset ja järjestelmän kuvaus} % Itse luvun otsikko. Huom ei numeroa!                                         santeri on ruma
\label{kuvaus} % Tähän kappaleeseen voi viitata \ref{kuvaus}
\thispagestyle{fancy} % Tarvitaan, jotta header/footer näkyvät otsikkosivuilla


\section{Mallintaminen}  % 3.1
    Tässä kappaleessa kuvaillaan kaksi käyttäjätarinaa. 


\subsection{Kayttotapauskaavio(t)}    % 3.1.1

\subsubsection{Käyttäjätapaus 1: Käyttäjätietojen tarkastelu}   % 3.1.1.1

    Antti Asiakas haluaa tarkastella itseään koskevia tietoja sekä raportteja. Antin tulee valita etusivulta löytyvä
    \textit{Omat tiedot} -painike, jotta hän pääsee katsomaan omia tietojaan. 

    Täältä Antti voi:

    \begin{itemize}
        \item tarkastella ja muuttaa omia yhteystietojaan
        \item tarkastella itseään koskevia raportteja
        \item poistaa itsensä järjestelmästä
        \item kirjautua ulos
    \end{itemize}

    Jos Antti valitsee itseään koskevat raportit, järjestelmä esittää hänelle hänen oman ostohistoriansa,
    voimassa olevat alennukset sekä mahdolliset suurostobonukset.
    Jos Antilla ei ole aiempaa historiaa Megantin kanssa, järjestelmä ei tarjoa hänelle raporttia.

\subsubsection{Käyttäjätapaus 2: Personoidun tarjouksen\/tajouksien luonti ja lähettäminen}     % 3.1.1.2

    Järjestelmä ehdottaa Mikko Myyntitykille mahdollista alennusta koskien tiettyä asiakasryhmää.
    Mahdollinen alennus perustuu järjestelmän omiin sisäisiin statistiikkoihin ja algoritmeihin.
    Mikko voi hyväksyä, olla hyväksymättä alennusta, tai halutessaan määritellä tarjouksen itse.
    Jos Mikko syöttää järjestelmään alennuksen, joka on erittäin suuri, järjestelmä antaa hänelle varoituksen.
    Mikko voi myös tarkastaa, että järjestelmä ei luo alennuksia, jotka ovat liian suuria tai pieniä.
    Alennusta myönnettäessä määritellään sen suuruus, kesto, tuoteryhmät sekä asiakaskohderyhmä.

    Jos jonkinlainen alennus myönnetään, järjestelmä lähettää asiakasryhmälle sähköisen ilmoituksen alennuksesta (tarjouskirje).

\subsection{Tietoyhteyskaavio(t)}   % 3.1.2


\subsection{Navigointikaavio}     % 3.1.3


\section{Käyttöliittymä}  % 3.2
    % Rautalankamallit neljästäkäyttöliittymänäkymästä (view) ja esimerkki mahdollisista ikkunoista (dialog). 
    % Pitää olla vähintään yleisnäkymä/päänäyttö, asiakkaan tietoihin liittyvä näkymä ja raportteihin liittyvä näkymä



\section{Vaatimukset}       % 3.3
    % Vaatimusten keruun perusteella täydennetty taulukko (alkuperäiset kehyskertomuksen mukaiset ja vaatimuksen keruussa löydetyt uudet) vaatimuksista, joko suoraan dokumenttiin tai liitteenä

    \subsection{Esimerkkivaatimus 1}
        % Tarkempi kuvaus toiminnallisesta vaatimuksesta, joka ei ole suoraan sidoksissa käyttäjän tekemisiin (esim joku automaattitoiminnoista)

    \subsection{Esimerkkivaatimus 2}
        % Tarkempi kuvaus ei-toiminnallisesta vaatimuksesta


    % Esimerkkivaatimuksen perusteella lukijan tulisi ymmärtää, mitä vaatimuksen perusteella 
    % järjestelmässä tapahtuu/mitä järjestelmä tekeeeli miten järjestelmän toiminta näkyy käyttäjille. 
    % Ei tarvitsemennä teknisiin ratkaisuihin (tietokantahaut jne)



\section{Ympäristö}     % 3.4
    \subsection{Liittyvät järjestelmät}     % 3.4.1
        % Mitä liittyviä järjestelmiä on (esim vanha järjestelmä, viranomaisjärjestelmät, muut yrityksen järjestelmät)? Mitä vaatimuksia ne asettavat määrittelydokumentin järjestelmälle?


    \subsection{Tarvittavat yhteydet ja muut ympäristön vaatimukset}  % 3.4.2
        % Ympäristön (toiminta-tai käyttöympäristö) asettamia vaatimuksia? Mitä vaaditaan järjestelmältä, mitä järjestelmä vaatii ympäristöltä.


\section{Jatkokehitysajatukset}     % 3.5

    Vaikka järjestelmä on tarkoitettu nykyisellä rakenteella Megantti konsernin käyttöön, niin sitä on kuitenkin tulevaisuudessa
    mahdollisuus jatkokehittää ja tuotteistaa, jolloin Megantti voisi tarjota sitä muille yrityksille valmiina CRM:änä.

    Myös, jos Megantti laajentaa toimintaa kansainvälisille markkinoille ja näille tytäryhtiöille otetaan myös käyttöön rakennettava
    CMR, tulee vähintään järjestelmän kieliä lisätä. Todennäköisesti myös uusia rajapintajärjestelmiä tulisi ottaa huomioon ja implementoida.


\section{Avoimet asiat}     % 3.6
    % Avoimeksi jääneitä asioita esim. määrittelyn aikataulun kiireen tai jonkin muun syyn takia.



   
