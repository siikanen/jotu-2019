\chapter{Johdanto} % Itse luvun otsikko. Huom ei numeroa!
\label{johdanto} % Tähän kappaleeseen voi viitata \ref{johdanto}
\thispagestyle{fancy} % Tarvitaan, jotta header/footer näkyvät otsikkosivuilla


Tämä dokumentti on luotu jäsentämään Megantti -konsernille tuotettavan asiakastietojärjestelmän kehitystyötä, sen vaiheita ja 
sisältöä. Dokumentti on tarkoitettu niin asiakkaalle, kuin järjestelmän kehitystiimille tarjoamaan tietoa ja dataa projektin
sisällöstä. Kyseessä on asiakastietojärjestelmä kodinkoneita ja elektroniikkaa myyvälle Megantti -konsernille, jonka myös 
konsernin samalla alalla toimivat tytäryhtiöt ottavat käyttöön. 

\section{Dokumentin tarkoitus ja sisältö} 

    Dokumentti kattaa sen ensimmäisessä vaiheessa tuotteen määrittelyn, tulevan järjestelmän käyttäjäryhmät ja käyttötarkoitukset
    ja projektin aikana käytettävän erikoissanaston sekä lyhenteet. Lisäksi ensimmäiseen vaiheeseen on kerätty ja analysoitu alustavia,
    järjestelmän lopullisen määrittelyyn ja kehittämiseen liittyviä sidosryhmiä ja vaatimuksia. Myös vaatimusten keruuta seuraava
    aikataulu ja suunnitelma löytyy dokumentista.

    Projektin toisessa vaiheessa dokumenttiin lisätään järjestelmää ja sen toimintaa kuvaavia kaavioita ja käyttöliittymäkuvia. 
    Lisäksi lopulliset vaatimukset täyden\-netään, kun määrittely on tehty mahdollisimman laajasti

    Toisen vaiheen lopuksi dokumentti kuvaa miten järjestelmää voidaan mahdollisesti jatkokehittää.
    Tässä dokumentissa ei määritellä, miten järjestelmän ylläpitovaihe tullaan toteuttamaan.


\section{Määriteltävä tuote, sen laajuus ja ympäristö}

    Dokumentissa määriteltävä tuote on Megantti -konsernin asiakastietojärjestelmä (CMR), jonka tarkoitus on kerätä kaikkien asiakkaiden asiakas-
    ja ostotiedot ja niiden mukaan analysoida asiakkaan toimintaa ja mm. tarjota hänelle kohdennettua mainontaa ja muita asiakasetuja, 
    sekä dataa omasta kuluttamisesta. Etenkin yritysasiakkaiden kohdalla järjestelmä tarjoaa erilaista dataa ja raportteja Megantin 
    kanssa tehdyistä kaupoista ja tarjouksista. 

    Järjestelmä tarjoaa Megantille (esim. myynti- ja markkinointiosastoille ja johtoportaalle) vastaavasti dataa ja raportteja heidän 
    suurimmista asiakkaistaan ja yleisistä myynti- ja markkinointitilastoista. Lisäksi konsernin analyytikot voivat tuottaa syvällisempiä 
    analyysejä esimerkiksi asiakkaiden kuluttajatottumuksista ja sitä kautta kohdentaa mainontaa ja tarjouskamppanjoita myynnin tehostamiseksi.

    Tuotettava asiakastietojärjestelmä tulee toimimaan sekä selaimessa että työpöytä\-sovelluksena. Lisäksi toteutetaan myös mobiiliaplikaatio.
    Näin voidaan varmistaa, että jokaiselle käyttäjäryhmälle löytyy parhaiten soveltuva päätelaite.


\section{Käyttäjät ja käyttötarkoitus}

    Tuotettava järjestelmä tulee olemaan erittäin monen erilasisen käyttäjän systeemi. Heidän osaamisensa ja järjestelmän käyttötarkoitukset eroavat suuresti
    toisistaan. Käyttäjät voidaan jakaa heidän asemansa perusteella neljään ryhmään:

    \begin{itemize}
        \item Ylläpitäjät (mm. järjestelmän hallinta ja rekisterin ylläpito)
        \item Sisällön tuottajat (mm. myynti- ja markkinointiosasto)
        \item Hyödyntäjät (mm. asiakaspalvelu, johtoryhmä)
        \item Asiakkaat (mm. yksityiset ja yritykset)
    \end{itemize}

    Koska rakennettava järjestelmä tulee toimimaan usean muun järjestelmän rajapinnassa ja hyvin monessa erilaisessa käyttöympäristössä,
    tulee sillä olemaan hyvin erilaisia käyttöympäristöjä. Esimerkiksi myynti- ja markkinointiosastot tulevat käyttämään järjestelmää 
    pääasiallisesti kiinteillä työpisteillä työpötäsovelluksena ja varaston toimijat mobiililaitteilla. Niin yritys- kuin yksityisasiakkaiden 
    käyttöympäristöt voivat olla hyvinkin vaihtelevia, joten on tärkeää, että jokaisessa ympäristössä käytettävyys ja ominaisuudet ovat yhtenevät.

    Käyttöliittymä tulee aluksi olemaan suomen- ja englanninkielinen, mutta lisäkäännöksiä voidaan tehdä, etenkin jos järjestelmä tuotteistetaan
    ja muut yritykset voivat ostaa sen käyttöönsä.

    \subsection*{Ylläpitäjät}

        Ylläpitäjien käyttäjäryhmä koostuu konsernin IT-osaston lohkosta, joka nimitetään ja tarvittaessa jatkokoulutetaan ylläpitämään ja hallinoimaan 
        järjestelmää. Tämän ryhmän käyttö on päivittäistä ja heidän oletetaan hallitsevan järjestelmän käyttö ja analysointi.

    \subsection*{Sisällön tuottajat}

        Sisällön tuottajiin lukeutuvat käytännössä kaikki järjestelmän pääkäyttäjät, eli myynti- ja markkinointiosasto sekä analyytikot, jotka kuuluvat 
        myös hyödyntäjiin. He käyttävät järjestelmää työssään päivittäin, ja etenkin analyytikoilta vaaditaan järjestelmän syvempää ymmärtämistä. Kuitenkaan ei
        voida olettaa, että tämän ryhmän yksittäinen käyttäjä osaisi käyttää järjestelmää vaadittavalla tasolla, jolloin voidaan tukeutua 
        ylläpitäjäryhmän antamaan tukeen. Ryhmä tulee käyttämään järjestelmää pääasiallisesti asiakkaan kanssa toimiakseen, joten etenkin hakutoimintoihin on heidän osaltaan kiinnitettävä
        huomiota.

    \subsection*{Hyödyntäjät}

        Järjestelmän hyödyntäjät ovat suurin käyttäjäryhmä, johon lukeutuvat muun muassa johtoryhmä, asiakaspalvelu, analyytikot, varastotoimijat sekä 
        ulkoiset logistiikkatoimijat. Tämän ryhmän käyttö vaihtelee päivittäisestä kuukausittaiseen, koska esimerkiksi johtoportaan henkilö saattaa katsoa 
        ainoastaan järjestelmän tuottamia kuukausiraportteja mobiililaitteilla, kun taas asiakaspalvelu suorittaa asiakastietojen hakuja päivittän.

        Etenkin tälle käyttäjäryhmälle on osaamisen vaihtelun kannalta tärkeää, että järjestelmän käyttöliittymä on tarvittavan selkeä, jotta 
        käyttö onnistuu nopeasti ja sujuvasti, eikä ylimääräisiä tukipyyntöjä käyttöä koskien tulisi IT-osastolle.

    \subsection*{Asiakkaat}

        Yksityisten asiakkaiden pääsääntöinen käyttöympäristö on selain. He pystyvät tarkastelemaan omia tietojaan ja yksinkertaisempia raportteja omista 
        ostotapahtumistaan. Lisäksi järjestelmä tarjoaa heille erilaisia tarjouksia, joita kohdennetaan asiakkaan tietojen mukaan.

        Varsinkin suuremmat yritysasiakkaat taas tulevat käyttämään heille paremmin räätälöityä työpöytäsovellusta, joka tuottaa yksityiskohtaisempaa
        dataa ja raportteja yrityksen sopimuksista ja kuluttamisesta Megantti -konsernin kanssa. Etenkin suurostot ja erilliset tilauspaketit 
        voidaan paremmin yksilöidä ja rakentaa juuri yrityksen tarpeisiin sopiviksi yhdessä myyntiosaston kanssa.


\section{Määritelmät, termit ja lyhenteet}

    Tässä luvussa esitellään projektin aikana esiintyviä määritelmiä väärinkäsityksiä aiheuttaville sanoille sekä kaikkien lyhenteiden merkitykset.

    \subsection{Lyhenteet}
        \begin{itemize}
		\item \gls{crm} = Customer Relationship Management = asiakastietojärjestelmä
            \item GDPR = General Data Protection Regulation = EU:n tietosuoja-asetus
            
        \end{itemize}

    \subsection{Määritelmät}
        \begin{itemize}
            \item Käyttäjä = Henkilö tai taho, joka jollain tavalla hyödyntää järjestelmää
            \item Ulkoinen käyttäjä = Yksityis- tai yritysasiakas
            \item Sisäinen käyttäjä = Kaikki konsernin sisällä toimivat tahot, jotka hyödyntävät järjestelmää. (esim. myynti- ja markkinointiosasto)
            \item Pääkäyttäjä(t) = Henkilö, joka käyttää järjestelmää säännöllisesti ja vastaa sen sisällön tuottamisesta.
            
        \end{itemize}

    

