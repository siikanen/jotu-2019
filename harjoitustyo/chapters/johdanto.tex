\chapter{Johdanto} % Itse luvun otsikko. Huom ei numeroa!
\label{johdanto} % Tähän kappaleeseen voi viitata \ref{johdanto}
\thispagestyle{fancy} % Tarvitaan, jotta header/footer näkyvät otsikkosivuilla

%\section{Dokumentin tarkoitus ja sisältö} % Ei peräkkäisiä otsikoita. (Tämä siis suoraan ison Johdanto otsikon alla)

    Tämä dokumentti on luotu jäsentämään Megantti -konsernille tuotettavan asiakastietojärjestelmän kehitystyötä, sen vaiheita ja 
    sisältöä. Dokumentti on tarkoitettu niin asiakkaalle, kuin järjestelmän kehitystiimille tarjoamaan tietoa ja dataa projektin
    sisällöstä. Kyseessä on asiakastietojärjestelmä kodinkoneita ja elektroniikkaa myyvälle Megantti -konsernille, jonka myös 
    konsernin samalla alalla toimivat tytäryhtiöt ottavat käyttöön. 

    Dokumentti kattaa sen ensimmäisessä vaiheessa tuotteen määrittelyn, tulevan järjestelmän käyttäjäryhmät ja käyttötarkoitukset
    ja projektin aikana käytettävän erikoissanaston sekä lyhenteet. Lisäksi ensimmäiseen vaiheeseen on kerätty ja analysoitu alustavia,
    järjestelmän lopullisen määrittelyyn ja kehittämiseen liittyviä sidosryhmiä ja vaatimuksia. Myös vaatimusten keruuta seuraava
    aikataulu ja suunnitelma löytyy dokumentista.

    Projektin toisessa vaiheessa dokumenttiin lisätään järjestelmää ja sen toimintaa kuvaavia kaavioita ja käyttöliittymäkuvia. 
    Lisäksi lopulliset vaatimukset täyden\-netään, kun määrittely on tehty mahdollisimman laajasti

    Toisen vaiheen lopuksi dokumentti kuvaa miten järjestelmää voidaan mahdollisesti jatkokehittää.
    Tässä dokumentissa ei määritellä, miten järjestelmän ylläpitovaihe tullaan toteuttamaan.


\section{Määriteltävä tuote, sen laajuus ja ympäristö}

    Dokumentissa määriteltävä tuote Megantti -konsernin asiakastietojärjestelmä, jonka tarkoitus on kerätä kaikkien asiakkaiden asiakas
    ja ostotiedot ja niiden mukaan analysoida asiakkaan toimintaa ja mm. tarjota hänelle kohdennettua mainontaa ja muita asiakasetuja, 
    kuin myös dataa omasta kuluttamisesta. Etenkin yritysasiakkaiden kohdalla järjestelmä tarjoaa erilaista dataa ja raportteja Megantin 
    kanssa tehdyistä kaupoista ja tarjouksista. 

    Järjestelmä tarjoaa Megantille (esim. myynti- ja markkinointiosastoille ja johtoportaalle) vastaavasti dataa ja raportteja heidän 
    suurimmista asiakkaistansa ja yleisistä myynti- ja markkinointitilastoista. Lisäksi konsernin analyytikot voivat tuottaa syvällisempiä 
    analyysejä esimerkiksi asiakkaiden kuluttajatottumuksista ja sitä kautta kohdentaa mainontaa ja tarjouskamppanioita myynnin tehostamiseksi.

    Tuotettava asiakastietojärjestelmä tulee toimimaan sekä selaimessa että työpöytä\-sovelluksena. Lisäksi toteutetaan myös mobiiliaplikaatio.
    Näin voidaan varmistaa, että jokaiselle käyttäjäryhmälle löytyy parhaiten soveltuva päätelaite.
