% !TEX TS-program = pdflatex
% !TEX encoding = UTF-8 Unicode

% This is a simple template for a LaTeX document using the "article" class.
% See "book", "report", "letter" for other types of document.
\PassOptionsToPackage{table}{xcolor}
\documentclass[12pt]{report} % use larger type; default would be 10pt

\usepackage[utf8]{inputenc} % set input encoding (not needed with XeLaTeX) 
\usepackage[finnish]{babel}
\usepackage[acronym,toc]{glossaries}


%%% Examples of Article customizations
% These packages are optional, depending whether you want the features they provide.
% See the LaTeX Companion or other references for full information.
\parindent0pt  \parskip10pt             % make block paragraphs

%%% PAGE DIMENSIONS
\usepackage{geometry} % to change the page dimensions
\geometry{a4paper} % or letterpaper (US) or a5paper or....
% \geometry{margin=2in} % for example, change the margins to 2 inches all round
% \geometry{landscape} % set up the page for landscape
% read geometry.pdf for detailed page layout information

\usepackage{graphicx} % support the \includegraphics command and options

% For rotating figures, tables, etc.
%  including their captions
\usepackage{rotating}
\usepackage{pdflscape}

% \usepackage[parfill]{parskip} % Activate to begin paragraphs with an empty line rather than an indent

%%% PACKAGES
\usepackage{booktabs} % for much better looking tables
\usepackage{array} % for better arrays (eg matrices) in maths
\usepackage{paralist} % very flexible & customisable lists (eg. enumerate/itemize, etc.)
\usepackage{verbatim} % adds environment for commenting out blocks of text & for better verbatim
\usepackage{subfig} % make it possible to include more than one captioned figure/table in a single float
\usepackage{float}
\usepackage{pgfgantt} % used to make gantt charts
% These packages are all incorporated in the memoir class to one degree or another...

% Custom colors for eg. links
\usepackage[table]{xcolor}
\definecolor{airforceblue}{rgb}{0.36, 0.54, 0.66}
\definecolor{coolblack}{rgb}{0.0, 0.18, 0.39}

%%% Make toc links clickable and hyperlinks available
\usepackage{hyperref}
\hypersetup{
    colorlinks,
    linktoc=all,
    citecolor=black,
    filecolor=black,
    linkcolor=coolblack,
    urlcolor=black
}

%%% HEADERS & FOOTERS
\usepackage{fancyhdr} % This should be set AFTER setting up the page geometry
\pagestyle{fancy} % options: empty , plain , fancy
%\renewcommand{\headrulewidth}{0pt} % customise the layout...


% Lorem ipsum dolor depsum...
\usepackage{lipsum}

% Fancy quotes
\usepackage{csquotes}

%%% SECTION TITLE APPEARANCE
%\usepackage{sectsty}
%\allsectionsfont{\sffamily\mdseries\upshape} % (See the fntguide.pdf for font help)
% (This matches ConTeXt defaults)


%%% TABLE STYLING
%\setlength{\arrayrulewidth}{1mm}
\usepackage{tabularx}
\setlength{\tabcolsep}{10pt}
\renewcommand{\arraystretch}{1.5}
\usepackage{colortbl}   
%%% ToC (table of contents) APPEARANCE
\usepackage[nottoc,notlof,notlot]{tocbibind} % Put the bibliography in the ToC
\usepackage[titles,subfigure]{tocloft} % Alter the style of the Table of Contents
\renewcommand{\cftsecfont}{\rmfamily\mdseries\upshape}
\renewcommand{\cftsecpagefont}{\rmfamily\mdseries\upshape} % No bold!


%\renewcommand{\chaptername}{Kappale}
%renewcommand{\contentsname}{Sisällys}
%\renewcommand{\tablename}{Taulukko}
%\renewcommand{\figurename}{Kuva}
\renewcommand{\bibname}{Lähteet}


%%% END Article customizations

\graphicspath{ {./images/} }

% Keywords command
\providecommand{\keywords}[1]
{
  \small	
  \textbf{\textit{Avainsanat ---}} #1
}

%%% The "real" document content comes below...

\title{TIE-02301 Johdatus ohjelmointiin: Harjoitustyö 1}
\author{Kujanen J., Palonen J., Siiranen S \& Werner, V}

%%% HEADERS & FOOTERS
\lhead{TIE-02301 Johdatus ohjelmointiin: Harjoitustyö 1}\chead{}\rhead{Versio \currentversion}
\lfoot{fffilename.pdf

%%% IMPORTANT! This file should ONLY contain the name on the output pdf on the first line.
}\cfoot{\thepage}\rfoot{\textit{Muokattu \today}}


%%% Make toc start at 0 instead of 1 by setting this to -1
\pagenumbering{roman}                   % roman page number for toc and versions
\setcounter{chapter}{0}

% Lyhenteet mukaan 
\setglossarystyle{altlist}
%
%%% LYHENTEET JA MÄÄRITELMÄT (+ käännökset?)
%
% Tässä tiedostossa määritellään tekstissä kätetyt lyhenteet
%
% LaTeX rakentaa näistä automaattisesti sivun, jossa esitellään kaikki
% dokumentissa esiintyvät lyhenteet (akronyymit). Valmiissa pdf:ssä
% jokainen lyhenne on myös linkki sen määritelmään, eli linkkiä 
% klikkaamalla pääsee suoraan sen määritelmään. 

\makeglossaries

    \newacronym{gdpr}{GDPR}{General Data Protection Regulation}
    \newacronym{crm}{CRM}{Customer Relationship Management}
    \newacronym{erp}{ERP}{Enterprise Resource Planning}
    \newacronym{ups}{UPS}{Uninterruptible Power Supply}
    \newacronym{api}{API}{Application Programming Interface}
    \newacronym{crud}{CRUD}{create, read, update, and delete}

    \newglossaryentry{gdprg}
    {
    name={General Data Protection Regulation (GDPR)},
	description={General Data Protection Regulation on Euroopan unionin asettama direktiivi tietosuojakäytännöistä},
	first={General Data Protection Regulation (GDPR)},
	text={GDPR}
    }
    \glsadd{gdprg}

    \newglossaryentry{crmg}
    {
	name={Customer Relationship Management (CRM)},
	description={Customer Relationship Management eli asiakkuudenhallintajärjestelmä \cite{crm, crm2}},
	first={Customer Relationship Management (CRM)},
	text={CRM}
    }
    \glsadd{crmg}

    \newglossaryentry{erpg}
    {
    name={Enterprise Resource Planning (ERP)},
    description={Toiminnanohjausjärjestelmä on yrityksen tietojärjestelmä, joka integroi eri toimintoja, esimerkiksi tuotantoa, jakelua, varastonhallintaa, laskutusta ja kirjanpitoa \cite{erp}},
    first={Enterprise Resource Planning (ERP)},
    text={ERP}
    }
    \glsadd{erpg}

    \newglossaryentry{upsg}
    {
    name={Uninterruptible Power Supply (UPS)},
    description={Järjestelmä, joka takaa tasaisen virransyötön, vaikka järjestelmän ulkopuolinen virransyöttö katkeaisi tai virransyötössä ilmenee joku muu häiriö. \cite{ups}},
    first={Uninterruptible Power Supply (UPS)},
    text={UPS}
    }
    \glsadd{upsg}

    \newglossaryentry{apig}
    {
    name={Application Programming Interface (API)},
    description={määritelmä, jonka mukaan eri ohjelmat voivat vaihtaa tietoja eli keskustella keskenään. \cite{api}},
    first={Application Programming Interface (API)},
    text={API}
    }
    \glsadd{apig}

    \newglossaryentry{crudg}
    {
    name={Create, Read, Update, and Delete (CRUD)},
    description={Tietojenkäsittelyn neljä perustoimintoa - luo, lue, päivitä ja poista},
    first={create, read, update, and delete (CRUD)},
    text={CRUD}
    }
    \glsadd{crudg}

    \newglossaryentry{kayttajag}
    {
    name={Käyttäjä},
    description={Henkilö tai taho, joka jollain tavalla hyödyntää järjestelmää},
    }
    \glsadd{kayttajag}

    \newglossaryentry{ulkoinenkayttajag}
    {
    name={Ulkoinen käyttäjä},
    description={Yksityis- tai yritysasiakas}
    }
    \glsadd{ulkoinenkayttajag}

    \newglossaryentry{sisainenkayttajag}
    {
    name={Sisäinen käyttäjä},
    description={Kaikki konsernin sisällä toimivat tahot, jotka hyödyntävät järjestelmää. (esim. myynti- ja markkinointiosasto)}
    }
    \glsadd{sisainenkayttajag}

    \newglossaryentry{paakayttajag}
    {
    name={Pääkäyttäjä},
    description={Käyttäjätili, jolla on korkein mahdollinen oikeustaso järjestelmään}
    }
    \glsadd{paakayttajag}

    \newglossaryentry{tietokantag}
    {
    name={Tietokanta},
    description={Tietojen keräämiseen ja järjestämiseen käytettävä työkalu, johon voidaan tallentaa esimerkiksi henkilöihin, tuotteisiin tai tilauksiin liittyviä tietoja \cite{kurssi}}
    }
    \glsadd{tietokantag}

    \newglossaryentry{redundantg}
    {
    name={Reduntant},
    description={Varmistettu (kahdennettu) linkki, joka toimii varana, jos ensisijainen järjestelmä kaatuu \cite{kurssi}}
    }
    \glsadd{redundantg}

    \newglossaryentry{virtualg}
    {
    name={Virtualisointi},
    description={Fyysisten laitteiden mallinnus virtuaaliseksi säilyttäen sama toiminnallisuus kuin alkuperäiselläkin laitteella.}
    }
    \glsadd{virtualg}

    \newglossaryentry{accesspolicyg}
    {
    name={Access Policy - Pääsynhallinta},
    description={Selvästi luokitellut ja rajatut käyttäjäryhmät, joille on tarkasti määritellyt oikeudet \cite{sommerville}}
    }
    \glsadd{accesspolicyg}

        




\begin{document}


    % Custom Titlepage, you may use \maketitle as well to use embedded values

    \begin{titlepage}
        \centering
        %\includegraphics[width=0.15\textwidth]{example-image-1x1}\par\vspace{1cm}
        {\scshape\LARGE Tampereen Teknillinen Yliopisto \par}
        \vspace{1cm}
        {\scshape\Large TIE-02301 Johdatus ohjelmointiin\par}
        \vspace{2.5cm}
        {\huge\bfseries Harjoitustyö 1 \par}
        %{\huge\bfseries Megantti konsernin asiakastietojärjestelmä\par}
        \vspace{4cm}
        {\hspace{2cm}\Large\itshape{Jussi} \textsc{Kujanen}, \textit{273161} \newline}
	    {\quad\textit{jussi.kujanen@tuni.fi }\par }
        {\hspace{2cm}\Large\itshape Joonas \textsc{Palonen}, \textit{272784}\newline}
	    {\quad \textit{joonas.palonen@tuni.fi} \par}
        {\hspace{2cm}\Large\itshape Santeri \textsc{Siiranen}, \textit{284281}\newline}
	    {\quad \textit{santeri.siiranen@tuni.fi} \par }
        {\hspace{2cm}\Large\itshape Ville \textsc{Werner}, \textit{273022}\newline}
	    {\quad \quad \textit{ville.werner@tuni.fi} \par}
        
        \vfill
%        Suorituspaikka: \textsc{???}
    
        \vfill
    
    % Bottom of the page
        {\large \today \par}
    \end{titlepage}

    
    %%% Versiohistoria
    \setcounter{page}{1}                    % make it start with "i"
    \chapter*{Versiohistoria}

%##################################
%# MUISTA PÄIVITTÄÄ VERSIONUMERO! #
%##################################
\newcommand{\currentversion}{0.1} % Vastattava viimeisintä päivitystä

\textbf{0.1} 16.02.2019 Dokumenttipohja luotu - Santeri Siiranen
\textbf{0.2} 18.02.2019 Lisätty sisältöä kappaleeseen "Johdanto"



    {
        \hypersetup{linkcolor=black}
        \tableofcontents  % Print table of contents
        \listoffigures
        \listoftables
    
    }
    


    %\printglossary[title=Lyhenteet, toctitle=Lyhenteet, type=\acronymtype]
       
    
    
    \begin{abstract}
        Tämä dokumentti on tarkoitettu havainnollistamaan Megantti-konsernille tuotettavan asiakastietojärjestelmän suunnittelua, kehitystä ja toteutusta. Dokumentissa pyritään olemaan mahdollisimman kattavia, jotta valmiin asiakastietojärjestelmän
        rakentaminen olisi mahdollismman vaivatonta suoraan tämän dokumentin pohjalta.

        Dokumentin eri luvuissa analysoimme muun muassa taustatilannetta, sidosryhmiä ja järjestelmän vaatimuksia.
        Luvussa \ref{keruu} analysoidaan syvemmin järjestelmän vaatimuksien keruuta. Luku \ref{kuvaus} puolestaan keskittyy enemmän järjestelmän vaatimuksiin, lopulliseen ulkoasuun ja toimintaperiaatteisiin.

        Kuvassa \ref{img:megantti} on Megantti-konsernin logo.
    
        
        
        \begin{figure}[H] % tämä H tarkoittaa tekstin mukana { h | t | b | p | H}
            \centering
            \includegraphics[width=5cm]{megantti.png}
            \caption{Megantti-konsernin logo} % Raportissa tulee olla selite jokaisen kuvan/kuvion/taulukon alla
            \label{img:megantti} % vaihda vastaamaan kuvaa, jolloin voit tekstissä viitata \ref{img:myimg}
        \end{figure}
        \mbox{}
        \vfill            
        \begin{center}      
                \keywords{asiakastietojärjestelmä, CRM, Megantti, ohjelmistotuotanto}
        \end{center}
    \end{abstract}

    
    % Älä lisää sisältöä tähän tiedostoon vaan niille varattuihin tiedostoihin
    
    \chapter{Johdanto} % Itse luvun otsikko. Huom ei numeroa!
\label{johdanto} % Tähän kappaleeseen voi viitata \ref{johdanto}
\thispagestyle{fancy} % Tarvitaan, jotta header/footer näkyvät otsikkosivuilla
\pagenumbering{arabic}                  % Start text with arabic 1



Tämä dokumentti on luotu jäsentämään Megantti -konsernille tuotettavan asiakastietojärjestelmän \cite{crm,crm2} kehitystyötä, sen vaiheita ja 
sisältöä. Dokumentti on tarkoitettu niin asiakkaalle, kuin järjestelmän kehitystiimille tarjoamaan tietoa ja dataa projektin
sisällöstä. Kyseessä on asiakastietojärjestelmä kodinkoneita ja elektroniikkaa myyvälle Megantti -konsernille, jonka myös 
konsernin samalla alalla toimivat tytäryhtiöt ottavat käyttöön. 

\section{Dokumentin tarkoitus ja sisältö} 

    Dokumentti kattaa sen ensimmäisessä vaiheessa tuotteen määrittelyn, tulevan järjestelmän käyttäjäryhmät ja käyttötarkoitukset
    ja projektin aikana käytettävän erikoissanaston sekä lyhenteet. Lisäksi ensimmäiseen vaiheeseen on kerätty ja analysoitu alustavia,
    järjestelmän lopullisen määrittelyyn ja kehittämiseen liittyviä sidosryhmiä ja vaatimuksia. Myös vaatimusten keruuta seuraava
    aikataulu ja suunnitelma löytyy dokumentista.

    Projektin toisessa vaiheessa dokumenttiin lisätään järjestelmää ja sen toimintaa kuvaavia kaavioita ja käyttöliittymäkuvia. 
    Lisäksi lopulliset vaatimukset täyden\-netään, kun määrittely on tehty mahdollisimman laajasti

    Toisen vaiheen lopuksi dokumentti kuvaa miten järjestelmää voidaan mahdollisesti jatkokehittää.
    Tässä dokumentissa ei määritellä, miten järjestelmän ylläpitovaihe tullaan toteuttamaan. \cite{kurssi}


\section{Määriteltävä tuote, sen laajuus ja ympäristö}

    Dokumentissa määriteltävä tuote on Megantti -konsernin asiakastietojärjestelmä (CMR), jonka tarkoitus on kerätä kaikkien asiakkaiden asiakas-
    ja ostotiedot ja niiden mukaan analysoida asiakkaan toimintaa ja mm. tarjota hänelle kohdennettua mainontaa ja muita asiakasetuja, 
    sekä dataa omasta kuluttamisesta. Etenkin yritysasiakkaiden kohdalla järjestelmä tarjoaa erilaista dataa ja raportteja Megantin 
    kanssa tehdyistä kaupoista ja tarjouksista. 

    Järjestelmä tarjoaa Megantille (esim. myynti- ja markkinointiosastoille ja johtoportaalle) vastaavasti dataa ja raportteja heidän 
    suurimmista asiakkaistaan ja yleisistä myynti- ja markkinointitilastoista. Lisäksi konsernin analyytikot voivat tuottaa syvällisempiä 
    analyysejä esimerkiksi asiakkaiden kuluttajatottumuksista ja sitä kautta kohdentaa mainontaa ja tarjouskamppanjoita myynnin tehostamiseksi.

    Tuotettava asiakastietojärjestelmä tulee toimimaan sekä selaimessa että työpöytä\-sovelluksena. Lisäksi toteutetaan myös mobiiliaplikaatio.
    Näin voidaan varmistaa, että jokaiselle käyttäjäryhmälle löytyy parhaiten soveltuva päätelaite.


\section{Käyttäjät ja käyttötarkoitus}

    Tuotettava järjestelmä tulee olemaan erittäin monen erilasisen käyttäjän systeemi. Heidän osaamisensa ja järjestelmän käyttötarkoitukset eroavat suuresti
    toisistaan. Käyttäjät voidaan jakaa heidän asemansa perusteella neljään ryhmään:

    \begin{itemize}
        \item Ylläpitäjät (mm. järjestelmän hallinta ja rekisterin ylläpito)
        \item Sisällön tuottajat (mm. myynti- ja markkinointiosasto)
        \item Hyödyntäjät (mm. asiakaspalvelu, johtoryhmä)
        \item Asiakkaat (mm. yksityiset ja yritykset)
    \end{itemize}

    Koska rakennettava järjestelmä tulee toimimaan usean muun järjestelmän raja\-pinnassa ja hyvin monessa erilaisessa käyttö\-ympäristössä,
    tulee sillä olemaan hyvin erilaisia käyttöympäristöjä. Esimerkiksi myynti- ja markkinointiosastot tulevat käyttämään järjestelmää 
    pääasiallisesti kiinteillä työpisteillä työpötäsovelluksena ja varaston toimijat mobiililaitteilla. Niin yritys- kuin yksityis\-asiakkaiden 
    käyttö\-ympä\-ris\-töt voivat olla hyvinkin vaihtelevia, joten on tärkeää, että jokaisessa ymp\-äris\-tössä käytet\-tävyys ja ominai\-suudet ovat yhtenevät.

    Käyttö\-liittymä tulee aluksi olemaan suomen- ja englanni\-nkielinen, mutta lisä\-kään\-nöksiä voidaan tehdä, etenkin jos järjes\-telmä tuot\-teiste\-taan
    ja muut yritykset voivat ostaa sen käyttöönsä.

    \subsection*{Ylläpitäjät}

        Ylläpitäjien käyttäjäryhmä koostuu kon\-sernin IT-osaston loh\-kosta, joka nimitetään ja tarvittaessa jatko\-koulutetaan yllä\-pitämään ja hal\-linoimaan 
        järjestelmää. Tämän ryhmän käyttö on päivittäistä ja heidän oletetaan hallitsevan järjestelmän käyttö ja analysointi.

    \subsection*{Sisällön tuottajat}

        Sisällön tuottajiin lukeutuvat käytännössä kaikki järjestelmän pääkäyttäjät, eli myynti- ja markkinointiosasto sekä analyytikot, jotka kuuluvat 
        myös hyödyntäjiin. He käyttävät järjestelmää työssään päivittäin, ja etenkin analyytikoilta vaaditaan järjestelmän syvempää ymmärtämistä. Kuitenkaan ei
        voida olettaa, että tämän ryhmän yksittäinen käyttäjä osaisi käyttää järjestelmää vaadittavalla tasolla, jolloin voidaan tukeutua 
        ylläpitäjäryhmän antamaan tukeen. Ryhmä tulee käyttämään järjestelmää pääasiallisesti asiakkaan kanssa toimiakseen, joten etenkin hakutoimintoihin on heidän osaltaan kiinnitettävä
        huomiota.

    \subsection*{Hyödyntäjät}

        Järjestelmän hyödyntäjät ovat suurin käyttäjäryhmä, johon lukeutuvat muun muassa johtoryhmä, asiakaspalvelu, analyytikot, varastotoimijat sekä 
        ulkoiset logistiikkatoimijat. Tämän ryhmän käyttö vaihtelee päivittäisestä kuukausittaiseen, koska esimerkiksi johtoportaan henkilö saattaa katsoa 
        ainoastaan järjestelmän tuottamia kuukausiraportteja mobiililaitteilla, kun taas asiakaspalvelu suorittaa asiakastietojen hakuja päivittän.

        Etenkin tälle käyttäjä\-ryhmälle on osaamisen vaihtelun kannalta tärkeää, että jär\-jes\-telmän käyttö\-liit\-tymä on tarvit\-tavan selkeä, jotta 
        käyttö onnistuu nopeasti ja sujuvasti, eikä ylimääräisiä tukipyyntöjä käyttöä koskien tulisi IT-osastolle.

    \subsection*{Asiakkaat}

        Yksityisten asiakkaiden pääsääntöinen käyttöympäristö on selain. He pystyvät tarkastelemaan omia tietojaan ja yksinkertaisempia raportteja omista 
        ostotapahtumistaan. Lisäksi järjestelmä tarjoaa heille erilaisia tarjouksia, joita kohdennetaan asiakkaan tietojen mukaan.

        Varsinkin suuremmat yritysasiakkaat taas tulevat käyttämään heille paremmin räätälöityä työpöytäsovellusta, joka tuottaa yksityiskohtaisempaa
        dataa ja raportteja yrityksen sopimuksista ja kuluttamisesta Megantti -konsernin kanssa. Etenkin suurostot ja erilliset tilauspaketit 
        voidaan paremmin yksilöidä ja rakentaa juuri yrityksen tarpeisiin sopiviksi yhdessä myyntiosaston kanssa.




    


    \chapter{Vaatimustenkeruusuunnitelma} % Itse luvun otsikko. Huom ei numeroa!
\label{keruu} % Tähän kappaleeseen voi viitata \ref{keruu}
\thispagestyle{fancy} % Tarvitaan, jotta header/footer näkyvät otsikkosivuilla

\lipsum[1]

\section{Otsikko}  % Kappaleen otsikko

% Täytetekstiä, poista kun lisäät oikeaa sisältöä
\lipsum

\subsection{Alaotsikko}

% Täytetekstiä, poista kun lisäät oikeaa sisältöä
\lipsum[1-3]


\section{Nykyisen dokumentaation analyysi}
Tässä kappaleessa käsitellään jo olemassa olevia dokumentaatioita, ja mitä vaatimuksia asiakastietojärjestelmälle voidaan niiden perusteella määrittää.



\subsection{Kehyskertomuksen analyysi}
Con-Salting Oy on kerännyt alustavia vaatimuksia kehyskertomukseen. Kehyskertomuksessa määritetään asiakastietojärjestelmälle alustavia vaatimuksia.
Kehyskertomuksen perusteella voidaan määrittää sidosryhmiä, jotka tulee ottaa huomioon, ja joiden kanssa tulee kommunikoida vaatimuksia määriteltäessä.
Tällaisia sidosryhmiä ovat muun muassa yrityksen johto ja myynti- ja markkinointiosasto.
sidosryhmät ovat esittäneet toivomuksina esimerkiksi asiakastietojärjestelmän nopean toiminnan, käytettävyyden erilaisilta päätelaitteilta sekä asiakkaan 
automaattisen profiloinnin. 
    Kehyskertomuksessa ilmenee myös muita asioita jotka tulee ottaa huomioon:
Järjestelmän tulee olla yhteensopiva muiden yrityksen järjestelmien kanssa kuten yrityksen varastojärjestelmän kanssa.
Asiakastietojärjestelmän tulee myös noudattaa GDPR:ää(General Data Protection Regulation)
Järjestelmän tulee pystyä hoitaa joitain asioita automaattisesti, kuten suurostobonukset.

Nämä vaatimukset ovat kuitenkin ainoastaan suuntaan antavia, ja niitä tulee tarkentaa vaatimustenkeruussa. 





\section{Vaatimustenkeruu metodit}
Vaatimustenkartutusmetodit voidaan jakaa kahteen eri osa-alueeseen: Epäsuoriin- ja suoriin metodeihin.
Tässä kappaleessa käsitellään Megantin asiakastietojärjestelmän kehitystä varten käytettäviä metodeja.


\subsection{Suorat metodit}
Suorissa kartutustekniikoissa järjestelmän vaatimuksia kartoitetaan yhdessä sidosryhmien kansssa.
Käytämme seuraavia suorakartutustekniikoita:

Haastattelut
Järjestämme kahdentyyppisiä haastatteluja sidosryhmille. Toisessa haastattelussa haastateltavat vastaavaat ennaltamääriteltyihin kysymyksiin. 
Toisessa haastattelussa haastattelu toteutetaan avoimesti. Haastateltaville esitetään erilaisia kysymyksiä järjestelmän toiminnalisuuteen liittyen jolloin
haastateltavien kanssa voidaan vuorovaikutteisesti pohtia millainen järjestelmän tulisi olla.

Havainnointi
Lähetämme 2 kehitystiimiläistä vierailemaan  Meganttiin. Nämä tiimiläiset seuraavat päivän ajan Megantin työntekijöiden työtä, ja tekevät 
muistiinpanoja työntekijöiden tarpeista järjestelmään liittyen. He havainnoivat myös mitä puutteita ja vahvuuksia nykyisessä asiakastietojärjestelmässä on.
Käytämme havainnointia siitä syystä, että ihmisten on usein vaikea pukea arkipäivän työntekoa sanoiksi. Havainnoin avulla pääsemme eroon tästä haittatekijästä.


Ryhmätapaamiset 
Projektin aikana järjestetään tapaamisia sidosryhmien kanssa. Näissä tapaamisissa kehitystiimi kertovat asiakastietojärjestelmän nykytilasta, kuvailevat tilanteita 
johon sidosryhmät voivat vastata kuinka he haluavat järjestelmän toimivan kyseisissä tilanteissa.


\subsection{Epäsuorat metodit}
Epäsuorissa kartutustekniikoissa sidosryhmiin ei olla suorassa kontaktissa, vaan käytetään jo ennalta olevia tietoja.
Käytämme seuraavia epäsuorakartutustekniikoita:

Taustatutkimus 
Con-Salting Oy on kerännyt jo muutamia yrityksen vaatimuksia kehyskertomukseen. 
Näitä vaatimuksia ovat esimerkiksi asiakastietojen ylläpito, ja asiakkaiden profilointi.
Vaatimukset ovat kuitenkin löysästi määriteltyjä ja vaativat tarkennusta.

Kyselyt
Osalle järjestelmän sidosryhmistä järjestetään kyselyitä, jossa selvitetään heidän mielestään tärkeimpiä järjestelmän vaatimuksia.
Tällaisia sidosryhmiä ovat sisäiset käyttäjät, kuten yrityksen johto, asiakaspalvelu ja järjestelmän ylläpito henkilöstö.

Prototyypit
Sisäisille käyttäjille valmistetaan kahteen otteeseen prototyyppi järjestelmästä. Käyttäjät voivat siis testata järjestelmää aikaisessa vaiheessa ja antaa palautetta.
    \chapter{Vaatimukset ja järjestelmän kuvaus} % Itse luvun otsikko. Huom ei numeroa!                                         santeri on ruma
\label{kuvaus} % Tähän kappaleeseen voi viitata \ref{kuvaus}
\thispagestyle{fancy} % Tarvitaan, jotta header/footer näkyvät otsikkosivuilla


\section{Mallintaminen}  % 3.1
    Tässä kappaleessa kuvaillaan kaksi käyttäjätarinaa. 


\subsection{Kayttotapauskaavio(t)}    % 3.1.1

\subsubsection{Käyttäjätapaus 1: Käyttäjätietojen tarkastelu}   % 3.1.1.1

    Antti Asiakas haluaa tarkastella itseään koskevia tietoja sekä raportteja. Antin tulee valita etusivulta löytyvä
    \textit{Omat tiedot} -painike, jotta hän pääsee katsomaan omia tietojaan. 

    Täältä Antti voi:

    \begin{itemize}
        \item tarkastella ja muuttaa omia yhteystietojaan
        \item tarkastella itseään koskevia raportteja
        \item poistaa itsensä järjestelmästä
        \item kirjautua ulos
    \end{itemize}

    Jos Antti valitsee itseään koskevat raportit, järjestelmä esittää hänelle hänen oman ostohistoriansa,
    voimassa olevat alennukset sekä mahdolliset suurostobonukset.
    Jos Antilla ei ole aiempaa historiaa Megantin kanssa, järjestelmä ei tarjoa hänelle raporttia.

\subsubsection{Käyttäjätapaus 2: Personoidun tarjouksen\/tajouksien luonti ja lähettäminen}     % 3.1.1.2

    Järjestelmä ehdottaa Mikko Myyntitykille mahdollista alennusta koskien tiettyä asiakasryhmää.
    Mahdollinen alennus perustuu järjestelmän omiin sisäisiin statistiikkoihin ja algoritmeihin.
    Mikko voi hyväksyä, olla hyväksymättä alennusta, tai halutessaan määritellä tarjouksen itse.
    Jos Mikko syöttää järjestelmään alennuksen, joka on erittäin suuri, järjestelmä antaa hänelle varoituksen.
    Mikko voi myös tarkastaa, että järjestelmä ei luo alennuksia, jotka ovat liian suuria tai pieniä.
    Alennusta myönnettäessä määritellään sen suuruus, kesto, tuoteryhmät sekä asiakaskohderyhmä.

    Jos jonkinlainen alennus myönnetään, järjestelmä lähettää asiakasryhmälle sähköisen ilmoituksen alennuksesta (tarjouskirje).

\subsection{Tietoyhteyskaavio(t)}   % 3.1.2


\subsection{Navigointikaavio}     % 3.1.3


\section{Käyttöliittymä}  % 3.2
    % Rautalankamallit neljästäkäyttöliittymänäkymästä (view) ja esimerkki mahdollisista ikkunoista (dialog). 
    % Pitää olla vähintään yleisnäkymä/päänäyttö, asiakkaan tietoihin liittyvä näkymä ja raportteihin liittyvä näkymä



\section{Vaatimukset}       % 3.3
    % Vaatimusten keruun perusteella täydennetty taulukko (alkuperäiset kehyskertomuksen mukaiset ja vaatimuksen keruussa löydetyt uudet) vaatimuksista, joko suoraan dokumenttiin tai liitteenä

    \subsection{Esimerkkivaatimus 1}
        % Tarkempi kuvaus toiminnallisesta vaatimuksesta, joka ei ole suoraan sidoksissa käyttäjän tekemisiin (esim joku automaattitoiminnoista)

    \subsection{Esimerkkivaatimus 2}
        % Tarkempi kuvaus ei-toiminnallisesta vaatimuksesta


    % Esimerkkivaatimuksen perusteella lukijan tulisi ymmärtää, mitä vaatimuksen perusteella 
    % järjestelmässä tapahtuu/mitä järjestelmä tekeeeli miten järjestelmän toiminta näkyy käyttäjille. 
    % Ei tarvitsemennä teknisiin ratkaisuihin (tietokantahaut jne)



\section{Ympäristö}     % 3.4
    \subsection{Liittyvät järjestelmät}     % 3.4.1
        % Mitä liittyviä järjestelmiä on (esim vanha järjestelmä, viranomaisjärjestelmät, muut yrityksen järjestelmät)? Mitä vaatimuksia ne asettavat määrittelydokumentin järjestelmälle?


    \subsection{Tarvittavat yhteydet ja muut ympäristön vaatimukset}  % 3.4.2
        % Ympäristön (toiminta-tai käyttöympäristö) asettamia vaatimuksia? Mitä vaaditaan järjestelmältä, mitä järjestelmä vaatii ympäristöltä.


\section{Jatkokehitysajatukset}     % 3.5

    Vaikka järjestelmä on tarkoitettu nykyisellä rakenteella Megantti konsernin käyttöön, niin sitä on kuitenkin tulevaisuudessa
    mahdollisuus jatkokehittää ja tuotteistaa, jolloin Megantti voisi tarjota sitä muille yrityksille valmiina CRM:änä.

    Myös, jos Megantti laajentaa toimintaa kansainvälisille markkinoille ja näille tytäryhtiöille otetaan myös käyttöön rakennettava
    CMR, tulee vähintään järjestelmän kieliä lisätä. Todennäköisesti myös uusia rajapintajärjestelmiä tulisi ottaa huomioon ja implementoida.


\section{Avoimet asiat}     % 3.6
    % Avoimeksi jääneitä asioita esim. määrittelyn aikataulun kiireen tai jonkin muun syyn takia.



   


    % LÄHTEET
    \chapter{Lahteet}
\label{lahteet}
\thispagestyle{fancy}

\textbf{1} Sommerville, Ian - Software engineering 10th edition (2016)
\textbf{2} https://en.wikipedia.org/wiki/Customer-relationship_management
\textbf{3} https://en.wikipedia.org/wiki/Enterprise_resource_planning
\textbf{4} kurssin luento- ja opetusmateriaali
\textbf{5} https://www.salesforce.com/eu/learning-centre/crm/what-is-crm/

    
    \printglossaries

    % LIITTEET
    \appendix
        
    \chapter{Aikatauluesimerkki}
\begin{sidewaysfigure}

\begin{ganttchart}{1}{28}
    % TODO Tästä saa vielä hienon kun vöhön käyttää aikaa hiomiseen. Atm karkea perusmalli
    \gantttitle{Neljän viikon aikajakso}{28} \\
    \gantttitlelist{1,...,28}{1} \\
    

	\ganttbar{Taustatutkimus}{1}{7} \\
	\ganttbar{Kyselyt}{3}{10} \\
	\ganttbar{Havainnointi}{3}{10} \\
	\ganttbar{Prototyyppi}{10}{15} \\
	\ganttbar{Haastattelut}{15}{22} \\
	\ganttbar{Ryhmätapaaminen}{15}{22} \\
	\ganttbar{Prototyyppi}{20}{24} \\

	\ganttlink{elem0}{elem1} 
	\ganttlink{elem0}{elem3} 
	\ganttlink{elem1}{elem3} 
	\ganttlink{elem2}{elem3} 
	\ganttlink{elem3}{elem4} 
	\ganttlink{elem4}{elem5} 
	\ganttlink{elem5}{elem6} 
	

\end{ganttchart}

\caption{Ehdotus eri menetelmien aikataulutuksesta Gantt\-kaavion muodossa}
\label{aikatauluesimerkki}

\end{sidewaysfigure}

    \chapter{Taulukko keskeisistä vaatimuksista}

\begin{landscape}
\begin{table}[]
\resizebox{21cm}{!}{% use resizebox with textwidth
    \begin{tabular}{llll}
    \multicolumn{4}{l}{Prioriteeti (1=Vähiten tärkeä, 2= Jonkin verran tärkeä, 3= Erittäin tärkeä)  Luokat (T=Tuottajat, H=Hyödyntäjät, J=Järjestelmävaatimus)}                                                                                                                                                                           \\ \hline
    \multicolumn{1}{|l|}{{\color[HTML]{000000} \textbf{Prioriteetti}}} & \multicolumn{1}{l|}{{\color[HTML]{000000} \textbf{Lähde}}} & \multicolumn{1}{l|}{{\color[HTML]{000000} \textbf{Luokka}}} & \multicolumn{1}{l|}{{\color[HTML]{000000} \textbf{Vaatimus}}}                                                \\ \hline
    \multicolumn{1}{|l|}{3}                                            & \multicolumn{1}{l|}{Haastattelut}                                      & \multicolumn{1}{l|}{T/H}                                    & \multicolumn{1}{l|}{Asiakastietojärjestelmää tulee pystyä käyttää eri päätelaitteilta.}                       \\ \hline
    \multicolumn{1}{|l|}{3}                                            & \multicolumn{1}{l|}{Taustatutkimus}                                      & \multicolumn{1}{l|}{J}                                    & \multicolumn{1}{l|}{Järjestelmän tulee noudattaa GDPR:ää.}                                                   \\ \hline
    \multicolumn{1}{|l|}{3}                                            & \multicolumn{1}{l|}{Taustatutkimus}                                      & \multicolumn{1}{l|}{J}                                      & \multicolumn{1}{l|}{Järjestelmän tulee olla yhteensopiva muiden olemassa olevien järjestelmien kanssa.}      \\ \hline
    \multicolumn{1}{|l|}{2}                                            & \multicolumn{1}{l|}{Havainnointi}                                      & \multicolumn{1}{l|}{T}                                      & \multicolumn{1}{l|}{Asiakkaiden tiedot tulee pystyä etsiä järjestelmästä nopeasti.}                           \\ \hline
    \multicolumn{1}{|l|}{3}                                            & \multicolumn{1}{l|}{Taustatutkimus}                                      & \multicolumn{1}{l|}{J}                                      & \multicolumn{1}{l|}{Järjestelmä rakennetaan olemassa olevan olevan ERP SQL constructoreiden päälle.}          \\ \hline
    \multicolumn{1}{|l|}{2}                                            & \multicolumn{1}{l|}{Prototyypit}                                      & \multicolumn{1}{l|}{T}                                      & \multicolumn{1}{l|}{Järjestelmän tulee seurata asiakkaiden ostoskäyttäytymistä.}                              \\ \hline
    \multicolumn{1}{|l|}{3}                                            & \multicolumn{1}{l|}{Havainnointi}                                      & \multicolumn{1}{l|}{T}                                      & \multicolumn{1}{l|}{Järjestelmän tulee pitää lokia kaikista tapahtumista.}                                   \\ \hline
    \multicolumn{1}{|l|}{2}                                            & \multicolumn{1}{l|}{Taustatutkimus}                                      & \multicolumn{1}{l|}{T}                                      & \multicolumn{1}{l|}{Järjestelmän pitää pystyä analysoida ja profiloida asiakkaita.}                         \\ \hline
    \multicolumn{1}{|l|}{1}                                            & \multicolumn{1}{l|}{Prototyypit/Haastattelut}                                      & \multicolumn{1}{l|}{T/H}                                    & \multicolumn{1}{l|}{Järjestelmän tulee olla helppokäyttöinen(Graphical user Interface.)}                    \\ \hline
    \multicolumn{1}{|l|}{1}                                            & \multicolumn{1}{l|}{Prototyypit}                                      & \multicolumn{1}{l|}{J}                                    & \multicolumn{1}{l|}{Järjestelmässä tulee olla monipuolisia toimintoja, kuten ostoshistoria, selainhistoria.}\\ \hline
    \multicolumn{1}{|l|}{2}                                            & \multicolumn{1}{l|}{Haastattelut}                                      & \multicolumn{1}{l|}{J}                                    & \multicolumn{1}{l|}{Järjestelmän luotettavuus tulee taata.}                                                 \\ \hline
    \multicolumn{1}{|l|}{1}                                            & \multicolumn{1}{l|}{Ryhmätapaamiset}                                      & \multicolumn{1}{l|}{H}                                    & \multicolumn{1}{l|}{Järjestelmän tulee pystyä yksilöllistää asiakkaan markkinointia.}                       \\ \hline
    \multicolumn{1}{|l|}{2}                                            & \multicolumn{1}{l|}{Ryhmätapaamiset}                                      & \multicolumn{1}{l|}{T}                                    & \multicolumn{1}{l|}{Järjestelmän tulee pitää kirjaa järjestelmätapahtumista.}                               \\ \hline
    \multicolumn{1}{|l|}{3}                                            & \multicolumn{1}{l|}{Taustatutkimus}                                      & \multicolumn{1}{l|}{J}                                    & \multicolumn{1}{l|}{Järjestelmän backendin tulee olla erotettu frontendistä (headless)}                               \\ \hline
    \multicolumn{1}{|l|}{3}                                            & \multicolumn{1}{l|}{Taustatutkimus}                                      & \multicolumn{1}{l|}{J}                                    & \multicolumn{1}{l|}{Järjestelmällä tulee olla web pohjainen skaalautuva käyttöliittymä}                               \\ \hline
   

    \end{tabular}
}
    \caption{Taulukko keskeisistä vaatimuksista}
    \label{tab:vaatimukset}
    \end{table}	
\end{landscape}

    \chapter{Taulukko lopullisista vaatimuksista}
\label{tab:vaatimukset2}

\begin{landscape}
\begin{table}[]
\resizebox{21cm}{!}{% use resizebox with textwidth
\begin{tabular}{lllll}
    \multicolumn{4}{l}{Prioriteeti (1=Vähiten tärkeä, 2= Jonkin verran tärkeä, 3= Erittäin tärkeä)  Luokat (T=Tuottajat, H=Hyödyntäjät, J=Järjestelmävaatimus)}                                                                                                                                                                           \\ \hline
    \multicolumn{1}{|l|}{1}		&			 \multicolumn{1}{|l|}{{\color[HTML]{000000} \textbf{Prioriteetti}}} & \multicolumn{1}{l|}{{\color[HTML]{000000} \textbf{Lähde}}} & \multicolumn{1}{l|}{{\color[HTML]{000000} \textbf{Luokka}}} & \multicolumn{1}{l|}{{\color[HTML]{000000} \textbf{Vaatimus}}}                                                \\ \hline
\multicolumn{1}{|l|}{2}		&			 \multicolumn{1}{|l|}{3}& \multicolumn{1}{l|}{Taustatutkimus}			& \multicolumn{1}{l|}{T/H}                                    & \multicolumn{1}{l|}{Järjestelmän tulee noudattaa GDPR:ää.}                                                   \\ \hline
\multicolumn{1}{|l|}{3}		&			 \multicolumn{1}{|l|}{3}& \multicolumn{1}{l|}{Haastattelut}				& \multicolumn{1}{l|}{J}                                    & \multicolumn{1}{l|}{Asiakastietojärjestelmää tulee pystyä käyttää eri päätelaitteilta.}                       \\ \hline
\multicolumn{1}{|l|}{4}		&			 \multicolumn{1}{|l|}{3}& \multicolumn{1}{l|}{Taustatutkimus}			& \multicolumn{1}{l|}{J}                                      & \multicolumn{1}{l|}{Järjestelmän tulee olla yhteensopiva muiden olemassa olevien järjestelmien kanssa.}      \\ \hline
\multicolumn{1}{|l|}{5}		&			 \multicolumn{1}{|l|}{2}& \multicolumn{1}{l|}{Havainnointi}				& \multicolumn{1}{l|}{T}                                      & \multicolumn{1}{l|}{Asiakkaiden tiedot tulee pystyä etsiä järjestelmästä nopeasti.}                           \\ \hline
\multicolumn{1}{|l|}{6}		&			 \multicolumn{1}{|l|}{3}& \multicolumn{1}{l|}{Haastattelut}				& \multicolumn{1}{l|}{J}                                      & \multicolumn{1}{l|}{Järjestelmä rakennetaan olemassa olevan olevan ERP SQL constructoreiden päälle.}          \\ \hline
\multicolumn{1}{|l|}{7}		&			 \multicolumn{1}{|l|}{2}& \multicolumn{1}{l|}{Prototyypit}				& \multicolumn{1}{l|}{T}                                      & \multicolumn{1}{l|}{Järjestelmän tulee seurata asiakkaiden ostoskäyttäytymistä.}                              \\ \hline
\multicolumn{1}{|l|}{8}		&			 \multicolumn{1}{|l|}{3}& \multicolumn{1}{l|}{Havainnointi}				& \multicolumn{1}{l|}{T}                                      & \multicolumn{1}{l|}{Järjestelmän tulee pitää lokia kaikista tapahtumista.}                                   \\ \hline
\multicolumn{1}{|l|}{9}		&			 \multicolumn{1}{|l|}{2}& \multicolumn{1}{l|}{Taustatutkimus}			& \multicolumn{1}{l|}{T}                                      & \multicolumn{1}{l|}{Järjestelmän pitää pystyä analysoida ja profiloida asiakkaita.}                         \\ \hline
\multicolumn{1}{|l|}{10}		&			 \multicolumn{1}{|l|}{1}& \multicolumn{1}{l|}{Prototyypit/Haastattelut}& \multicolumn{1}{l|}{T/H}                                    & \multicolumn{1}{l|}{Järjestelmän tulee olla helppokäyttöinen(Graphical user Interface.)}                    \\ \hline
\multicolumn{1}{|l|}{11}		&			 \multicolumn{1}{|l|}{1}& \multicolumn{1}{l|}{Prototyypit}			& \multicolumn{1}{l|}{J}                                    & \multicolumn{1}{l|}{Järjestelmässä tulee olla monipuolisia toimintoja, kuten ostoshistoria, selainhistoria.}\\ \hline
\multicolumn{1}{|l|}{12}		&			 \multicolumn{1}{|l|}{2}& \multicolumn{1}{l|}{Haastattelut}			& \multicolumn{1}{l|}{J}                                    & \multicolumn{1}{l|}{Järjestelmän luotettavuus tulee taata.}                                                 \\ \hline
\multicolumn{1}{|l|}{13}		&			 \multicolumn{1}{|l|}{1}& \multicolumn{1}{l|}{Ryhmätapaamiset}	& \multicolumn{1}{l|}{H}                                    & \multicolumn{1}{l|}{Järjestelmän tulee pystyä yksilöllistää asiakkaan markkinointia.}                       \\ \hline
    \multicolumn{1}{|l|}{14}		&			 \multicolumn{1}{|l|}{2}& \multicolumn{1}{l|}{Ryhmätapaamiset}	& \multicolumn{1}{l|}{T}                                    & \multicolumn{1}{l|}{Järjestelmän tulee pitää kirjaa järjestelmätapahtumista.}                               \\ \hline
    \multicolumn{1}{|l|}{15}		&			 \multicolumn{1}{|l|}{2}& \multicolumn{1}{l|}{Taustatutkimus}	& \multicolumn{1}{l|}{J}                                    & \multicolumn{1}{l|}{Järjestelmän backendin tulee olla erotettu frontendistä (headless)}                               \\ \hline
    \multicolumn{1}{|l|}{16}		&			 \multicolumn{1}{|l|}{3}& \multicolumn{1}{l|}{Haastattelut}	& \multicolumn{1}{l|}{J}                                    & \multicolumn{1}{l|}{Järjestelmällä tulee olla web pohjainen skaalautuva käyttöliittymä}                               \\ \hline
    \multicolumn{1}{|l|}{17}		&			 \multicolumn{1}{|l|}{3}& \multicolumn{1}{l|}{Ryhmätapaamiset}	& \multicolumn{1}{l|}{H}                                    & \multicolumn{1}{l|}{Asiakkaan tulee osata käyttää järjestelmää ilman alustavaa koulutusta}                               \\ \hline
    \multicolumn{1}{|l|}{18}		&			 \multicolumn{1}{|l|}{3}& \multicolumn{1}{l|}{Prototyypit}	& \multicolumn{1}{l|}{J}                                    & \multicolumn{1}{l|}{Headless frontin API:n tulee käyttää HSTS menetelmää}																	\\ \hline
    \multicolumn{1}{|l|}{19}		&			 \multicolumn{1}{|l|}{3}& \multicolumn{1}{l|}{Taustatutkimus}	& \multicolumn{1}{l|}{H}                                    & \multicolumn{1}{l|}{Asiakastietojärjestelmän on automatisoitava varmuuskopiointi}								\\ \hline
    \multicolumn{1}{|l|}{20}		&			 \multicolumn{1}{|l|}{2}& \multicolumn{1}{l|}{Ryhmätapaamiset}	& \multicolumn{1}{l|}{J}                                    & \multicolumn{1}{l|}{Asiakastietojärjestelmästä on oltava dokumnentaatio}														\\ \hline
    \multicolumn{1}{|l|}{21}		&			 \multicolumn{1}{|l|}{2}& \multicolumn{1}{l|}{Taustatutkimus}	& \multicolumn{1}{l|}{J}                                    & \multicolumn{1}{l|}{Järjestelmän tietoja tulee pystyä editoimaan CRUD-menetelmän mukaisesti, jokaisen clientin kautta}									\\ \hline
    \multicolumn{1}{|l|}{22}		&			 \multicolumn{1}{|l|}{3}& \multicolumn{1}{l|}{Taustatutkimus}	& \multicolumn{1}{l|}{J}                                    & \multicolumn{1}{l|}{Järjestelmässä tulee olla selkeä \textit{Access policy} käyttäjille}						\\ \hline
    \multicolumn{1}{|l|}{23}		&			 \multicolumn{1}{|l|}{2}& \multicolumn{1}{l|}{Taustatutkimus}	& \multicolumn{1}{l|}{H}                                    & \multicolumn{1}{l|}{Asiakas tulee pystyä etsiä järjestelmästä mahdollisimman monen tietoyhteyden avulla}						\\ \hline
                         
    


\end{tabular}
}
    \caption{Taulukko lopullisista vaatimuksista}
    \end{table}	
\end{landscape}
    \chapter{Sidosryhmätaulukko}

% Please add the following required packages to your document preamble:
% \usepackage[table,xcdraw]{xcolor}
% If you use beamer only pass "xcolor=table" option, i.e. \documentclass[xcolor=table]{beamer}
\begin{landscape}
\begin{table}[]
\label{tab:sidosryhmataulukko}
\resizebox{21cm}{!}{% use resizebox with textwidth
\begin{tabular}{llllllllllllllll}
Sidosryhmäanalyysin tulokset                                                                                                                       & V1.0                                                                                                                &                                                                                                                  &                                                                                                &                                                                                    &                                  &                                   &                                        &                                                                                         &                                      &                                                       &                                      &                                                                                                    &                                     &                                   &                                 \\
                                                                                                                                                   &                                                                                                                     &                                                                                                                  &                                                                                                &                                                                                    &                                  &                                   &                                        &                                                                                         &                                      &                                                       &                                      &                                                                                                    &                                     &                                   &                                 \\
\rowcolor[HTML]{9B9B9B} 
{\color[HTML]{CB0000} \begin{tabular}[c]{@{}l@{}}Sidosryhmäluokka\\   (laajempi sidosryhmäluokka, esim. käyttäjä tai\\   hyödyntäjä)\end{tabular}} & {\color[HTML]{CB0000} \begin{tabular}[c]{@{}l@{}}Sidosryhmä (esim. IT-osasto,\\   markkinointiosasto)\end{tabular}} & {\color[HTML]{CB0000} \begin{tabular}[c]{@{}l@{}}Sidosryhmän oikeutus (Miksi se on\\   osallinen?)\end{tabular}} & {\color[HTML]{3531FF} \begin{tabular}[c]{@{}l@{}}Tarpeellinen\\   osallistuminen\end{tabular}} & {\color[HTML]{3531FF} Tiedonkeruumenetelmä}                                        & {\color[HTML]{3531FF} Rajapinta} & {\color[HTML]{3531FF} Tavoitteet} & {\color[HTML]{3531FF} Toiminnallisuus} & {\color[HTML]{3531FF} \begin{tabular}[c]{@{}l@{}}Tekniset\\   rajoitukset\end{tabular}} & {\color[HTML]{3531FF} Käyttökokemus} & {\color[HTML]{3531FF} Käytettävyys  (cross platform)} & {\color[HTML]{3531FF} Siirrettävyys} & {\color[HTML]{3531FF} \begin{tabular}[c]{@{}l@{}}Liiketoiminnalliset\\   rajoitukset\end{tabular}} & {\color[HTML]{3531FF} Suorituskyky} & {\color[HTML]{3531FF} Tietoturva} & {\color[HTML]{3531FF} Ylläpito} \\
                                                                                                                                                   &                                                                                                                     &                                                                                                                  &                                                                                                &                                                                                    &                                  &                                   &                                        &                                                                                         &                                      &                                                       &                                      &                                                                                                    &                                     &                                   &                                 \\
\begin{tabular}[c]{@{}l@{}}Operatiivinen\\   toiminta\end{tabular}                                                                                 &                                                                                                                     &                                                                                                                  &                                                                                                &                                                                                    &                                  &                                   &                                        &                                                                                         &                                      &                                                       &                                      &                                                                                                    &                                     &                                   & x                               \\
\begin{tabular}[c]{@{}l@{}}Rekisterien\\   ylläpito\end{tabular}                                                                                   & IT-osasto                                                                                                           & Sisällönhaltija                                                                                                  & \begin{tabular}[c]{@{}l@{}}Määrittely \&\\   käyttöönotto\end{tabular}                         & \begin{tabular}[c]{@{}l@{}}haastattelut,\\   ryhmätapaamiset, protot\end{tabular}  &                                  &                                   & x                                      &                                                                                         &                                      &                                                       &                                      &                                                                                                    &                                     &                                   & x                               \\
\begin{tabular}[c]{@{}l@{}}Järjestelmän\\   hallinta\end{tabular}                                                                                  & IT-osasto                                                                                                           & Pääkäyttäjä                                                                                                      & \begin{tabular}[c]{@{}l@{}}Määrittely \&\\   käyttöönotto\end{tabular}                         & \begin{tabular}[c]{@{}l@{}}haastattelut,\\   ryhmätapaamiset, protot\end{tabular}  &                                  &                                   & x                                      & x                                                                                       &                                      &                                                       &                                      &                                                                                                    &                                     &                                   &                                 \\
                                                                                                                                                   &                                                                                                                     &                                                                                                                  &                                                                                                &                                                                                    &                                  &                                   &                                        &                                                                                         &                                      &                                                       &                                      &                                                                                                    &                                     &                                   &                                 \\
\begin{tabular}[c]{@{}l@{}}Pääasialliset\\   käyttäjät\end{tabular}                                                                                &                                                                                                                     &                                                                                                                  &                                                                                                &                                                                                    &                                  &                                   &                                        &                                                                                         &                                      &                                                       &                                      &                                                                                                    &                                     &                                   &                                 \\
\begin{tabular}[c]{@{}l@{}}Käyttäjä /\\   hyödyntäjä\end{tabular}                                                                                  & Markkinointiosasto                                                                                                  & Päähyödyntäjä                                                                                                    & Määrittely                                                                                     & \begin{tabular}[c]{@{}l@{}}ryhmätapaamiset,\\   havainnointi, protot\end{tabular}  &                                  &                                   & x                                      &                                                                                         & x                                    & x                                                     &                                      &                                                                                                    &                                     &                                   &                                 \\
\begin{tabular}[c]{@{}l@{}}Käyttäjä /\\   hyödyntäjä\end{tabular}                                                                                  & Myyntiosasto                                                                                                        & Päähyödyntäjä                                                                                                    & Määrittely                                                                                     & \begin{tabular}[c]{@{}l@{}}ryhmätapaamiset,\\   havainnointi, protot\end{tabular}  &                                  &                                   & x                                      &                                                                                         & x                                    & x                                                     &                                      &                                                                                                    &                                     &                                   &                                 \\
Hyödyntäjä                                                                                                                                         & Analyytikot                                                                                                         & Hyödyntäjä                                                                                                       & Määrittely                                                                                     & haastattelut, kyselyt                                                              &                                  &                                   & x                                      &                                                                                         &                                      &                                                       &                                      &                                                                                                    &                                     &                                   &                                 \\
\begin{tabular}[c]{@{}l@{}}Käyttäjä /\\   hyödyntäjä\end{tabular}                                                                                  & Asiakaspalvelu                                                                                                      & Hyödyntäjä                                                                                                       & Määrittely                                                                                     & haastattelut, kyselyt                                                              &                                  &                                   &                                        &                                                                                         & x                                    &                                                       &                                      &                                                                                                    &                                     &                                   &                                 \\
                                                                                                                                                   &                                                                                                                     &                                                                                                                  &                                                                                                &                                                                                    &                                  &                                   &                                        &                                                                                         &                                      &                                                       &                                      &                                                                                                    &                                     &                                   &                                 \\
\begin{tabular}[c]{@{}l@{}}Ulkoiset\\   käyttäjät\end{tabular}                                                                                     &                                                                                                                     &                                                                                                                  &                                                                                                &                                                                                    &                                  &                                   &                                        &                                                                                         &                                      &                                                       &                                      &                                                                                                    &                                     &                                   &                                 \\
Hyödyntäjä                                                                                                                                         & Henkilöasiakas                                                                                                      & Mahdollinen käyttäjä                                                                                             & Määrittely                                                                                     & taustatutkimus                                                                     &                                  &                                   &                                        &                                                                                         &                                      & x                                                     &                                      & x                                                                                                  &                                     &                                   &                                 \\
Hyödyntäjä                                                                                                                                         & Yritykset                                                                                                           & Mahdollinen käyttäjä                                                                                             & Määrittely                                                                                     & taustatutkimus                                                                     &                                  &                                   &                                        &                                                                                         &                                      & x                                                     &                                      & x                                                                                                  &                                     &                                   &                                 \\
                                                                                                                                                   &                                                                                                                     &                                                                                                                  &                                                                                                &                                                                                    &                                  &                                   &                                        &                                                                                         &                                      &                                                       &                                      &                                                                                                    &                                     &                                   &                                 \\
\begin{tabular}[c]{@{}l@{}}Ympäristön\\   tahot\end{tabular}                                                                                       &                                                                                                                     &                                                                                                                  &                                                                                                &                                                                                    &                                  &                                   &                                        &                                                                                         &                                      &                                                       &                                      &                                                                                                    &                                     &                                   &                                 \\
\begin{tabular}[c]{@{}l@{}}Muut\\   järjestelmät\end{tabular}                                                                                      & Muut järjestelmät                                                                                                   & Rajapinta                                                                                                        & Koko projekti                                                                                  & \begin{tabular}[c]{@{}l@{}}taustatutkimus,\\   protot\end{tabular}                 & x                                &                                   & x                                      & x                                                                                       &                                      & x                                                     & x                                    &                                                                                                    &                                     & x                                 &                                 \\
\begin{tabular}[c]{@{}l@{}}Muut tahot /\\   vaikuttaja\end{tabular}                                                                                & \begin{tabular}[c]{@{}l@{}}Tietosuojavaltuuttettu\\   (GDPR)\end{tabular}                                           & Vaikuttaja                                                                                                       & Määrittely                                                                                     & taustatutkimus                                                                     &                                  &                                   &                                        & x                                                                                       &                                      &                                                       &                                      & x                                                                                                  &                                     & x                                 &                                 \\
\begin{tabular}[c]{@{}l@{}}Muut tahot /\\   vaikuttaja\end{tabular}                                                                                & Kilpailuviranomainen                                                                                                & Vaikuttaja                                                                                                       & Määrittely                                                                                     & taustatutkimus                                                                     &                                  &                                   &                                        &                                                                                         &                                      &                                                       &                                      & x                                                                                                  &                                     &                                   &                                 \\
\begin{tabular}[c]{@{}l@{}}Muut tahot /\\   vaikuttaja\end{tabular}                                                                                & Veroviranomainen                                                                                                    & Vaikuttaja                                                                                                       & Määrittely                                                                                     & taustatutkimus                                                                     &                                  &                                   &                                        &                                                                                         &                                      &                                                       &                                      & x                                                                                                  &                                     &                                   &                                 \\
\begin{tabular}[c]{@{}l@{}}Muut tahot /\\   vaikuttaja\end{tabular}                                                                                & Kuluttajaviranomainen                                                                                               & Vaikuttaja                                                                                                       & Määrittely                                                                                     & taustatutkimus                                                                     &                                  &                                   &                                        &                                                                                         &                                      &                                                       &                                      & x                                                                                                  &                                     &                                   &                                 \\
Vaikuttaja                                                                                                                                         & Rahoittaja                                                                                                          & Vaikuttaja                                                                                                       & Koko projekti                                                                                  & \begin{tabular}[c]{@{}l@{}}ryhmätapaamiset,\\   protot\end{tabular}                &                                  & x                                 &                                        &                                                                                         &                                      &                                                       &                                      &                                                                                                    &                                     &                                   &                                 \\
Hyödyntäjä                                                                                                                                         & Varastotoimijat                                                                                                     & \begin{tabular}[c]{@{}l@{}}Hyödyntäjä /\\   rajapinta\end{tabular}                                               & Määrittely                                                                                     & taustatutkimus                                                                     & x                                &                                   &                                        &                                                                                         &                                      &                                                       &                                      &                                                                                                    &                                     &                                   &                                 \\
\begin{tabular}[c]{@{}l@{}}Hyödyntäjä /\\   alihankkija\end{tabular}                                                                               & Logistiikka                                                                                                         & Rajapinta                                                                                                        & Määrittely                                                                                     & taustatutkimus                                                                     & x                                &                                   & x                                      &                                                                                         &                                      &                                                       & x                                    &                                                                                                    &                                     &                                   &                                 \\
                                                                                                                                                   &                                                                                                                     &                                                                                                                  &                                                                                                &                                                                                    &                                  &                                   &                                        &                                                                                         &                                      &                                                       &                                      &                                                                                                    &                                     &                                   &                                 \\
Ydinprojektiryhmä                                                                                                                                  &                                                                                                                     &                                                                                                                  &                                                                                                &                                                                                    &                                  &                                   &                                        &                                                                                         &                                      &                                                       &                                      &                                                                                                    &                                     &                                   &                                 \\
Omistaja                                                                                                                                           & Megantti                                                                                                            & Omistaja                                                                                                         & \begin{tabular}[c]{@{}l@{}}Määrittely \&\\   käyttöönotto\end{tabular}                         & \begin{tabular}[c]{@{}l@{}}ryhmätapaamiset,\\   kyselyt\end{tabular}               &                                  & x                                 &                                        &                                                                                         &                                      &                                                       &                                      &                                                                                                    &                                     &                                   &                                 \\
\begin{tabular}[c]{@{}l@{}}Toimeksiantaja\\   / hyödyntäjä\end{tabular}                                                                            & Johtoryhmä                                                                                                          & Omistaja / hyödyntäjä                                                                                            & \begin{tabular}[c]{@{}l@{}}Määrittely \&\\   käyttöönotto\end{tabular}                         & \begin{tabular}[c]{@{}l@{}}ryhmätapaamiset,\\   kyselyt, havainnointi\end{tabular} &                                  & x                                 & x                                      &                                                                                         &                                      &                                                       &                                      &                                                                                                    &                                     &                                   &                                 \\
Muut tahot                                                                                                                                         & Projektipäällikkö                                                                                                   & Toteuttaja                                                                                                       & Koko projekti                                                                                  &                                                                                    &                                  & x                                 &                                        & x                                                                                       &                                      &                                                       &                                      &                                                                                                    & x                                   &                                   &                                 \\
                                                                                                                                                   &                                                                                                                     &                                                                                                                  &                                                                                                &                                                                                    &                                  &                                   &                                        &                                                                                         &                                      &                                                       &                                      &                                                                                                    &                                     &                                   &                                 \\
\begin{tabular}[c]{@{}l@{}}Projektin\\   toteuttajat\end{tabular}                                                                                  &                                                                                                                     &                                                                                                                  &                                                                                                &                                                                                    &                                  &                                   &                                        &                                                                                         &                                      &                                                       &                                      &                                                                                                    &                                     &                                   &                                 \\
Tuottaja                                                                                                                                           & Ohjelmistotuottajat                                                                                                 & Toteuttaja                                                                                                       & Koko projekti                                                                                  &                                                                                    &                                  &                                   & x                                      &                                                                                         &                                      &                                                       &                                      &                                                                                                    &                                     &                                   &                                 \\
Tuottaja                                                                                                                                           & Ohjelmistotestaajat                                                                                                 & Toteuttaja                                                                                                       & Koko projekti                                                                                  &                                                                                    &                                  &                                   & x                                      &                                                                                         &                                      &                                                       &                                      &                                                                                                    &                                     &                                   &                                 \\
                                                                                                                                                   &                                                                                                                     &                                                                                                                  &                                                                                                &                                                                                    &                                  &                                   &                                        &                                                                                         &                                      &                                                       &                                      &                                                                                                    &                                     &                                   &                                 \\
Kilpailijat                                                                                                                                        & \begin{tabular}[c]{@{}l@{}}esim. Gigantti ja\\   Verkkokauppa.com\end{tabular}                                      & kilpailija                                                                                                       & Määrittely                                                                                     & taustatutkimus                                                                     &                                  &                                   &                                        &                                                                                         &                                      &                                                       &                                      & x                                                                                                  &                                     &                                   &                                 \\
                                                                                                                                                   &                                                                                                                     &                                                                                                                  &                                                                                                &                                                                                    &                                  &                                   &                                        &                                                                                         &                                      &                                                       &                                      &                                                                                                    &                                     &                                   &                                 \\
                                                                                                                                                   &                                                                                                                     &                                                                                                                  &                                                                                                &                                                                                    &                                  &                                   &                                        &                                                                                         &                                      &                                                       &                                      &                                                                                                    &                                     &                                   &                                
\end{tabular}
}
\end{table}
\end{landscape}


    \chapter{Käyttöliittymäesimerkit}
\label{kayttoliittyma}

\begin{figure}
    \includegraphics[width=\textwidth]{gui/asiakas.png}
    \caption{Asiakkaan päänäkymä}
    \label{img:asiakas}
\end{figure}

\begin{figure}
    \includegraphics[width=\textwidth]{gui/sisainen_paa.png}
    \caption{Sisäisen käyttäjän päänäkymä}
    \label{img:sisainenpaa}
\end{figure}

\begin{figure}
    \includegraphics[width=\textwidth]{gui/sisainen.png}
    \caption{Sisäisen käyttäjän näkymä tilausten tarkasteluun}
    \label{img:sisainen}
\end{figure}

\begin{figure}
    \includegraphics[width=\textwidth]{gui/yllapito.png}
    \caption{Ylläpidon päänäkymä}
    \label{img:yllapito}
\end{figure}

    \chapter{Sisäisen toimijan käyttötapauskaavio}
\label{kayttotapaus2}

\begin{figure}
    \includegraphics[width=\textwidth]{kayttotapauskaavio2.pdf}
    \caption{Sisäisen toimijan käyttötapauskaavio}
    \label{img:asiakas}
\end{figure}

    

\end{document}
