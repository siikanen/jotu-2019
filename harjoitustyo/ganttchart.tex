\chapter{Aikatauluesimerkki}
\begin{sidewaysfigure}

	\begin{ganttchart}{1}{28}[
	    x unit  = 3.3cm	
		]
    % TODO Tästä saa vielä hienon kun vöhön käyttää aikaa hiomiseen. Atm karkea perusmalli
    \gantttitle{Neljän viikon aikajakso}{28} \\
    \gantttitlelist{1,...,28}{1} \\
    
\ganttgroup{Epäsuorat metodit}{1}{13} \\
	\ganttbar{Taustatutkimus}{1}{10} \\
	\ganttbar{Kyselyt}{3}{10} \\
	\ganttbar{Prototyyppi 1}{10}{13} \\
	\ganttmilestone{Prototyyppi 1 valmis}{14} \\

\ganttgroup{Suorat metodit}{15}{28} \\
	\ganttbar{Haastattelut}{15}{22} \\
	\ganttbar{Ryhmätapaaminen}{15}{22} \\
	\ganttbar{Prototyyppi 2}{20}{28} \\
	\ganttbar{Havainnointi}{3}{18} \\

	% linkit / nuolet

	% epäsuorat
	\ganttlink{elem1}{elem4} \\
	\ganttlink{elem2}{elem4} \\
	\ganttlink{elem3}{elem4} \\
	
	% Suorat metodit osa
	\ganttlink{elem6}{elem8} \\
	\ganttlink{elem7}{elem8} \\

\end{ganttchart}

\caption{Ehdotus eri menetelmien aikataulutuksesta Gantt\-kaavion muodossa}
\label{aikatauluesimerkki}

\end{sidewaysfigure}
