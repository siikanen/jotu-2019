\chapter{Aikatauluesimerkki}
\begin{sidewaysfigure}

\begin{ganttchart}{1}{28}
	\label{aikatauluesimerkki}
    % TODO Tästä saa vielä hienon kun vöhön käyttää aikaa hiomiseen. Atm karkea perusmalli
    \gantttitle{Neljän viikon aikajakso}{28} \\
    \gantttitlelist{1,...,28}{1} \\
    

	\ganttbar{Taustatutkimus}{1}{7} \\
	\ganttbar{Kyselyt}{3}{10} \\
	\ganttbar{Havainnointi}{3}{10} \\
	\ganttbar{Prototyyppi}{10}{15} \\
	\ganttbar{Haastattelut}{15}{22} \\
	\ganttbar{Ryhmätapaaminen}{15}{22} \\
	\ganttbar{Prototyyppi}{20}{24} \\

	\ganttlink{elem0}{elem1} 
	\ganttlink{elem0}{elem3} 
	\ganttlink{elem1}{elem3} 
	\ganttlink{elem2}{elem3} 
	\ganttlink{elem3}{elem4} 
	\ganttlink{elem4}{elem5} 
	\ganttlink{elem5}{elem6} 
	

	\caption{Ehdotus eri menetelmien aikataulutuksesta Gantt-kaavion muodossa}
\end{ganttchart}

\end{sidewaysfigure}
