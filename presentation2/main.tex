\documentclass{beamer}
%
% Choose how your presentation looks.
%
% For more themes, color themes and font themes, see:
% http://deic.uab.es/~iblanes/beamer_gallery/index_by_theme.html
%
\mode<presentation>
{
  \usetheme{Berkeley}      % or try Darmstadt, Madrid, Warsaw, ...
  \usecolortheme{dove} % or try albatross, beaver, crane, ...
  \usefonttheme{default}  % or try serif, structurebold, ...
  \setbeamertemplate{navigation symbols}{}
  \setbeamertemplate{caption}[numbered]
} 

\usepackage[finnish]{babel}
\usepackage[utf8x]{inputenc}
\usepackage{pgfgantt}
\usepackage{graphicx}

\title[JOTU-2019 Megantti]{Megantti konsernin asiakastietojärjestelmä}
\author{Ryhmä 39}
\institute{Tampereen Teknillinen Yliopisto}
\date{\today}

\begin{document}

\begin{frame}
  \titlepage
\end{frame}

% Uncomment these lines for an automatically generated outline.
\begin{frame}{Tänään käsittelemme}
	\tableofcontents[pausesections]
\end{frame}

\section{Johdanto}

\begin{frame}{Esittely}
\begin{itemize}
  \item<1-> Me olemme
  \item<2-> Megantti on kansallinen kodinkoneita ja elektroniikkaa myyvä konserni
\end{itemize}

\begin{figure}
\includegraphics[scale=1.0]{../megantti.png}
	\caption{Megantti konsernin logo}
\end{figure}

\end{frame}

\section{Työn edistyminen}


\subsection{Määriteltävä järjestelmä}

\begin{frame}{Millainen on hyvä CRM?}

Hyvän CRM piirteitä

	\begin{itemize}
		\item<2-> Helppokäyttöinen
		\item<3-> Nopea
		\item<4-> Yhteensopiva
		\item<5-> Luotettava
	\end{itemize}
\end{frame}

\subsection{Sidosryhmäanalyysin tulokset}

\begin{frame}{Sidosyhmät}
\begin{itemize}
	\item Sisäiset käyttäjät
	\item Ulkoiset käyttäjät
	\item Järjestelmän hallinta
	\item Tuottajat (Myynti)
	\item Viranomaiset ja lainsäädäntö
	\item Kilpailijat
\end{itemize}
\end{frame}


\subsection{Alustavat vaatimukset}

\begin{frame}{Alustavat vaatimukset}

\begin{table}[]
\resizebox{\textwidth}{!}{% use resizebox with textwidth
    \begin{tabular}{llll}
    \multicolumn{4}{l}{Prioriteeti(1=Vähiten tärkeä, 2= Jonkin verran tärkeä, 3= Erittäin tärkeä)  Luokat(T=Tuottajat, H=Hyödyntäjät)}                                                                                                                                                                           \\ \hline
    \multicolumn{1}{|l|}{{\color[HTML]{000000} \textbf{Prioriteetti}}} & \multicolumn{1}{l|}{{\color[HTML]{000000} \textbf{Lähde}}} & \multicolumn{1}{l|}{{\color[HTML]{000000} \textbf{Luokka}}} & \multicolumn{1}{l|}{{\color[HTML]{000000} \textbf{Vaatimus}}}                                                \\ \hline
    \multicolumn{1}{|l|}{3}                                            & \multicolumn{1}{l|}{Haastattelut}                                      & \multicolumn{1}{l|}{T/H}                                    & \multicolumn{1}{l|}{Asiakastietojärjestelmää tulee pystyä käyttää eri päätelaitteilta.}                       \\ \hline
    \multicolumn{1}{|l|}{3}                                            & \multicolumn{1}{l|}{Taustatutkimus}                                      & \multicolumn{1}{l|}{T/H}                                    & \multicolumn{1}{l|}{Järjestelmän tulee noudattaa GDPR:ää.}                                                   \\ \hline
    \multicolumn{1}{|l|}{3}                                            & \multicolumn{1}{l|}{Taustatutkimus}                                      & \multicolumn{1}{l|}{T}                                      & \multicolumn{1}{l|}{Järjestelmän tulee olla yhteensopiva muiden olemassa olevien järjestelmien kanssa.}      \\ \hline
    \multicolumn{1}{|l|}{2}                                            & \multicolumn{1}{l|}{Havainnointi}                                      & \multicolumn{1}{l|}{T}                                      & \multicolumn{1}{l|}{Asiakkaiden tiedot tulee pystyä etsiä järjestelmästä nopeasti.}                           \\ \hline
    \multicolumn{1}{|l|}{3}                                            & \multicolumn{1}{l|}{Haastattelut}                                      & \multicolumn{1}{l|}{T}                                      & \multicolumn{1}{l|}{Järjestelmä rakennetaan olemassa olevan olevan ERP SQL constructoreiden päälle.}          \\ \hline
    \multicolumn{1}{|l|}{2}                                            & \multicolumn{1}{l|}{Prototyypit}                                      & \multicolumn{1}{l|}{T}                                      & \multicolumn{1}{l|}{Järjestelmän tulee seurata asiakkaiden ostoskäyttäytymistä.}                              \\ \hline
    \multicolumn{1}{|l|}{3}                                            & \multicolumn{1}{l|}{Havainnointi}                                      & \multicolumn{1}{l|}{T}                                      & \multicolumn{1}{l|}{Järjestelmän tulee pitää lokia kaikista tapahtumista.}                                   \\ \hline
    \multicolumn{1}{|l|}{2}                                            & \multicolumn{1}{l|}{Taustatutkimus}                                      & \multicolumn{1}{l|}{T}                                      & \multicolumn{1}{l|}{Järjestelmän pitää pystyä analysoida ja profiloida asiakkaita.}                         \\ \hline
    \multicolumn{1}{|l|}{1}                                            & \multicolumn{1}{l|}{Prototyypit/Haastattelut}                                      & \multicolumn{1}{l|}{T/H}                                    & \multicolumn{1}{l|}{Järjestelmän tulee olla helppokäyttöinen(Graphical user Interface.)}                    \\ \hline
    \multicolumn{1}{|l|}{1}                                            & \multicolumn{1}{l|}{Prototyypit}                                      & \multicolumn{1}{l|}{T/H}                                    & \multicolumn{1}{l|}{Järjestelmässä tulee olla monipuolisia toimintoja, kuten ostoshistoria, selainhistoria.}\\ \hline
    \multicolumn{1}{|l|}{2}                                            & \multicolumn{1}{l|}{Haastattelut}                                      & \multicolumn{1}{l|}{T/H}                                    & \multicolumn{1}{l|}{Järjestelmän luotettavuus tulee taata.}                                                 \\ \hline
    \multicolumn{1}{|l|}{1}                                            & \multicolumn{1}{l|}{Ryhmätapaamiset}                                      & \multicolumn{1}{l|}{H}                                    & \multicolumn{1}{l|}{Järjestelmän tulee pystyä yksilöllistää asiakkaan markkinointia.}                       \\ \hline
    \multicolumn{1}{|l|}{2}                                            & \multicolumn{1}{l|}{Ryhmätapaamiset}                                      & \multicolumn{1}{l|}{T}                                    & \multicolumn{1}{l|}{Järjestelmän tulee pitää kirjaa järjestelmätapahtumista.}                               \\ \hline
   

    \end{tabular}
}
    \caption{Taulukko keskeisistä vaatimuksista}
    \label{tab:vaatimukset}
    \end{table}	
\end{frame}

\subsection{Työnjako}

\begin{frame}{Miten toimimme}

	\begin{itemize}
		\item <2-> Haastattelut
		\item <3-> Havainnointi
		\item <4-> Ryhmätapaamiset
	\end{itemize}

\onslide<5-> Lisäksi

	\begin{itemize}
		\item<6-> Taustatutkimus
		\item<7-> Kyselyt
		\item<8-> Prototyypit
	\end{itemize}

\end{frame}


\section{Aikataulu\-suunnitelma}


\begin{frame}{Aikataulusuunnitelma}

\begin{figure}
\resizebox{\textwidth}{!}{% use resizebox with textwidth
\begin{ganttchart}{1}{28}
    \gantttitle{Neljän viikon aikajakso}{28} \\
    \gantttitlelist{1,...,28}{1} \\
    
\ganttgroup{Epäsuorat metodit}{1}{13} \\
	\ganttbar{Taustatutkimus}{1}{10} \\
	\ganttbar{Kyselyt}{3}{10} \\
	\ganttbar{Prototyyppi 1}{10}{13} \\
	\ganttmilestone{Prototyyppi 1 valmis}{14} \\

\ganttgroup{Suorat metodit}{15}{28} \\
	\ganttbar{Haastattelut}{15}{22} \\
	\ganttbar{Ryhmätapaaminen}{15}{22} \\
	\ganttbar{Prototyyppi 2}{20}{28} \\
	\ganttbar{Havainnointi}{3}{18} \\

	% linkit / nuolet

	% epäsuorat
	\ganttlink{elem1}{elem4} \\
	\ganttlink{elem2}{elem4} \\
	\ganttlink{elem3}{elem4} \\
	
	% Suorat metodit osa
	\ganttlink{elem6}{elem8} \\
	\ganttlink{elem7}{elem8} \\

\end{ganttchart}
}
\end{figure}
\end{frame}

\begin{frame}

\end{frame}

\end{document}
